\documentclass[ignorenonframetext,aspectratio=169]{beamer}
\setbeamertemplate{caption}[numbered]
\setbeamertemplate{caption label separator}{: }
\setbeamercolor{caption name}{fg=normal text.fg}
\beamertemplatenavigationsymbolsempty
\usepackage{lmodern}
\usepackage{amssymb,amsmath}
\usepackage{ifxetex,ifluatex}
\usepackage{fixltx2e} % provides \textsubscript
\ifnum 0\ifxetex 1\fi\ifluatex 1\fi=0 % if pdftex
  \usepackage[T1]{fontenc}
  \usepackage[utf8]{inputenc}
\else % if luatex or xelatex
  \ifxetex
    \usepackage{mathspec}
  \else
    \usepackage{fontspec}
  \fi
  \defaultfontfeatures{Ligatures=TeX,Scale=MatchLowercase}
\fi
% use upquote if available, for straight quotes in verbatim environments
\IfFileExists{upquote.sty}{\usepackage{upquote}}{}
% use microtype if available
\IfFileExists{microtype.sty}{%
\usepackage{microtype}
\UseMicrotypeSet[protrusion]{basicmath} % disable protrusion for tt fonts
}{}
\newif\ifbibliography
\hypersetup{
            pdftitle={Extracting information from FX Options},
            pdfauthor={Jorge A. Chan-Lau},
            pdfborder={0 0 0},
            breaklinks=true}
\urlstyle{same}  % don't use monospace font for urls
\usepackage{color}
\usepackage{fancyvrb}
\newcommand{\VerbBar}{|}
\newcommand{\VERB}{\Verb[commandchars=\\\{\}]}
\DefineVerbatimEnvironment{Highlighting}{Verbatim}{commandchars=\\\{\}}
% Add ',fontsize=\small' for more characters per line
\newenvironment{Shaded}{}{}
\newcommand{\KeywordTok}[1]{\textcolor[rgb]{0.00,0.44,0.13}{\textbf{#1}}}
\newcommand{\DataTypeTok}[1]{\textcolor[rgb]{0.56,0.13,0.00}{#1}}
\newcommand{\DecValTok}[1]{\textcolor[rgb]{0.25,0.63,0.44}{#1}}
\newcommand{\BaseNTok}[1]{\textcolor[rgb]{0.25,0.63,0.44}{#1}}
\newcommand{\FloatTok}[1]{\textcolor[rgb]{0.25,0.63,0.44}{#1}}
\newcommand{\ConstantTok}[1]{\textcolor[rgb]{0.53,0.00,0.00}{#1}}
\newcommand{\CharTok}[1]{\textcolor[rgb]{0.25,0.44,0.63}{#1}}
\newcommand{\SpecialCharTok}[1]{\textcolor[rgb]{0.25,0.44,0.63}{#1}}
\newcommand{\StringTok}[1]{\textcolor[rgb]{0.25,0.44,0.63}{#1}}
\newcommand{\VerbatimStringTok}[1]{\textcolor[rgb]{0.25,0.44,0.63}{#1}}
\newcommand{\SpecialStringTok}[1]{\textcolor[rgb]{0.73,0.40,0.53}{#1}}
\newcommand{\ImportTok}[1]{#1}
\newcommand{\CommentTok}[1]{\textcolor[rgb]{0.38,0.63,0.69}{\textit{#1}}}
\newcommand{\DocumentationTok}[1]{\textcolor[rgb]{0.73,0.13,0.13}{\textit{#1}}}
\newcommand{\AnnotationTok}[1]{\textcolor[rgb]{0.38,0.63,0.69}{\textbf{\textit{#1}}}}
\newcommand{\CommentVarTok}[1]{\textcolor[rgb]{0.38,0.63,0.69}{\textbf{\textit{#1}}}}
\newcommand{\OtherTok}[1]{\textcolor[rgb]{0.00,0.44,0.13}{#1}}
\newcommand{\FunctionTok}[1]{\textcolor[rgb]{0.02,0.16,0.49}{#1}}
\newcommand{\VariableTok}[1]{\textcolor[rgb]{0.10,0.09,0.49}{#1}}
\newcommand{\ControlFlowTok}[1]{\textcolor[rgb]{0.00,0.44,0.13}{\textbf{#1}}}
\newcommand{\OperatorTok}[1]{\textcolor[rgb]{0.40,0.40,0.40}{#1}}
\newcommand{\BuiltInTok}[1]{#1}
\newcommand{\ExtensionTok}[1]{#1}
\newcommand{\PreprocessorTok}[1]{\textcolor[rgb]{0.74,0.48,0.00}{#1}}
\newcommand{\AttributeTok}[1]{\textcolor[rgb]{0.49,0.56,0.16}{#1}}
\newcommand{\RegionMarkerTok}[1]{#1}
\newcommand{\InformationTok}[1]{\textcolor[rgb]{0.38,0.63,0.69}{\textbf{\textit{#1}}}}
\newcommand{\WarningTok}[1]{\textcolor[rgb]{0.38,0.63,0.69}{\textbf{\textit{#1}}}}
\newcommand{\AlertTok}[1]{\textcolor[rgb]{1.00,0.00,0.00}{\textbf{#1}}}
\newcommand{\ErrorTok}[1]{\textcolor[rgb]{1.00,0.00,0.00}{\textbf{#1}}}
\newcommand{\NormalTok}[1]{#1}

% Prevent slide breaks in the middle of a paragraph:
\widowpenalties 1 10000
\raggedbottom

\AtBeginPart{
  \let\insertpartnumber\relax
  \let\partname\relax
  \frame{\partpage}
}
\AtBeginSection{
  \ifbibliography
  \else
    \let\insertsectionnumber\relax
    \let\sectionname\relax
    \frame{\sectionpage}
  \fi
}
\AtBeginSubsection{
  \let\insertsubsectionnumber\relax
  \let\subsectionname\relax
  \frame{\subsectionpage}
}

\setlength{\parindent}{0pt}
\setlength{\parskip}{6pt plus 2pt minus 1pt}
\setlength{\emergencystretch}{3em}  % prevent overfull lines
\providecommand{\tightlist}{%
  \setlength{\itemsep}{0pt}\setlength{\parskip}{0pt}}
\setcounter{secnumdepth}{0}

\title{Extracting information from FX Options}
\author{Jorge A. Chan-Lau}
\date{February 7, 2018}

\begin{document}
\frame{\titlepage}

\begin{frame}{Objectives}

\begin{itemize}
\tightlist
\item
  We want to build this
\end{itemize}

\begin{center}\includegraphics[width=0.6\linewidth]{images/figJPYRND} \end{center}

\end{frame}

\begin{frame}[fragile]{What we will cover}

\begin{itemize}
\tightlist
\item
  Learn about basic FX structures
\item
  Develop some intuition about FX option prices
\item
  Get comfortable with a variety of methods
\item
  Side benefit: learn \texttt{R} and \texttt{RStudio}!
\item
  Lecture notes available at\\
  \color{red} \url{https://jchanlauimf.github.io/IMF_FXOptions}
\end{itemize}

\end{frame}

\begin{frame}{}

\color{blue} \LARGE{Part A:}\\
\LARGE{The Very Basics}

\end{frame}

\begin{frame}{Basics: a plain vanilla call option}

\begin{center}\includegraphics[width=0.8\linewidth]{images/figCallOptionMaturity} \end{center}

\end{frame}

\begin{frame}{Basics: a plain vanilla put option}

\begin{center}\includegraphics[width=0.8\linewidth]{images/figPutOptionMaturity} \end{center}

\end{frame}

\begin{frame}{Market reality}

\begin{itemize}
\tightlist
\item
  Price quote available for

  \begin{itemize}
  \tightlist
  \item
    ATM plain options
  \end{itemize}
\item
  Price quotes available for \textbf{FX structures}

  \begin{itemize}
  \tightlist
  \item
    risk reversals
  \item
    butterfly spreads
  \end{itemize}
\item
  Main data sources

  \begin{itemize}
  \tightlist
  \item
    Bloomberg
  \item
    Reuters
  \item
    Investment banks' data portals
  \end{itemize}
\end{itemize}

\end{frame}

\begin{frame}{}

\color{blue} \LARGE{Part B:}\\
\LARGE{Eyeballing the Data}

\end{frame}

\begin{frame}[fragile]{Getting the data in (1)}

\begin{enumerate}
\def\labelenumi{\arabic{enumi}.}
\tightlist
\item
  Start \texttt{RStudio}
\item
  Set the working directory to where we placed the data file
\item
  Issue this command in the console (replace with your directory)
\end{enumerate}

\begin{Shaded}
\begin{Highlighting}[]
\NormalTok{my_wdir =}\StringTok{ "D:/IMF_FXCourse"}  \CommentTok{# replace with your directory}
\KeywordTok{setwd}\NormalTok{(my_wdir)}
\end{Highlighting}
\end{Shaded}

\end{frame}

\begin{frame}[fragile]{Getting the data in (2)}

Clean memory, set up the needed libraries

\begin{Shaded}
\begin{Highlighting}[]
\KeywordTok{rm}\NormalTok{(}\DataTypeTok{list=}\KeywordTok{ls}\NormalTok{())             }\CommentTok{# Clean up memory}
\KeywordTok{library}\NormalTok{(ggplot2)          }\CommentTok{# Graphic library}
\KeywordTok{library}\NormalTok{(lubridate)        }\CommentTok{# Date manipulation library}
\KeywordTok{library}\NormalTok{(dplyr)            }\CommentTok{# Data manipulation library}
\KeywordTok{source}\NormalTok{(}\StringTok{"auxFunctions.R"}\NormalTok{)  }\CommentTok{# Auxiliary functions}
\end{Highlighting}
\end{Shaded}

\end{frame}

\begin{frame}[fragile]{Getting the data in (3)}

\begin{Shaded}
\begin{Highlighting}[]
\NormalTok{filename =}\StringTok{ "2018_IET_Options_data.csv"}
\NormalTok{data =}\StringTok{ }\KeywordTok{read.csv}\NormalTok{(filename, }\DataTypeTok{header=}\OtherTok{TRUE}\NormalTok{)}
\NormalTok{data}\OperatorTok{$}\NormalTok{Dates =}\StringTok{ }\KeywordTok{mdy_hm}\NormalTok{(}\KeywordTok{as.character}\NormalTok{(data}\OperatorTok{$}\NormalTok{Dates))}
\end{Highlighting}
\end{Shaded}

\begin{center}\includegraphics[width=1\linewidth]{images/figDataStructure} \end{center}

\end{frame}

\begin{frame}{The data (1)}

FX related

\begin{itemize}
\tightlist
\item
  Spot: the GBPUSD spot exchange rate, i.e.~USD per GBP
\item
  FWD3M: the 3-month GBPUSD forward exchange rate
\item
  Rf: the 3-month GBP money market deposit rate, annualized (in percent)
\item
  Rd: the 3-month USD money market deposit rate, annualized (in percent)
\end{itemize}

\end{frame}

\begin{frame}{The data (2)}

FX option related

\begin{itemize}
\tightlist
\item
  ATM: the at-the-money implied volatility of a GBPUSD option with
  strike price equal to ATM
\item
  RR25D3M: the price of a 25\(\Delta\) risk reversal, in annualized
  volatility units (in percent)
\item
  BF25D3M: the price of a 25\(\Delta\) butterfly spread, in annualized
  volatility units (in percent)
\item
  RR10D3M: the price of a 10\(\Delta\) risk reversal, in annualized
  volatility units (in percent)
\item
  BF10D3M: the price of a 10\(\Delta\) butterfly spread, in annualized
  volatility units (in percent)
\end{itemize}

\end{frame}

\begin{frame}{The data (3)}

\[ \text{Implied domestic rate from covered interest parity:  }F = S \frac{\exp \left(\text{Implied } R_d \times T\right)}{\exp \left(R_f \times T\right)} \]

\begin{center}\includegraphics[width=0.6\linewidth]{images/figSyntheticUSDLoan} \end{center}

\end{frame}

\begin{frame}[fragile]{What the data tell us: spot and forward FX rates
(1)}

\begin{Shaded}
\begin{Highlighting}[]
\CommentTok{# Plot of the spot exchange rate with vertical line at Brexit vote}
\NormalTok{p1 =}\StringTok{ }\KeywordTok{ggplot}\NormalTok{(data, }\KeywordTok{aes}\NormalTok{(Dates, Spot)) }\OperatorTok{+}\StringTok{ }\KeywordTok{geom_line}\NormalTok{(}\DataTypeTok{colour=}\StringTok{"blue"}\NormalTok{) }
\NormalTok{p1 =}\StringTok{ }\NormalTok{p1 }\OperatorTok{+}\StringTok{ }\KeywordTok{geom_vline}\NormalTok{(}\DataTypeTok{xintercept=}\KeywordTok{as.POSIXct}\NormalTok{(}\KeywordTok{as.Date}\NormalTok{((}\StringTok{"2016-06-22 UTC"}\NormalTok{))),}
                     \DataTypeTok{linetype=}\StringTok{"longdash"}\NormalTok{)}
\NormalTok{p1}
\end{Highlighting}
\end{Shaded}

\begin{figure}
\includegraphics[width=1\linewidth]{2018_02_07_IMF_FXCourse_files/figure-beamer/unnamed-chunk-8-1} \caption{USDGBP spot exchange rate}\label{fig:unnamed-chunk-8}
\end{figure}

\end{frame}

\begin{frame}[fragile]{What the data tell us: spot and forward FX rates
(2)}

\begin{Shaded}
\begin{Highlighting}[]
\NormalTok{p2 =}\StringTok{ }\KeywordTok{ggplot}\NormalTok{(data, }\KeywordTok{aes}\NormalTok{(Dates, FWD3M)) }\OperatorTok{+}\StringTok{ }\KeywordTok{geom_line}\NormalTok{(}\DataTypeTok{colour=}\StringTok{"red"}\NormalTok{) }
\NormalTok{p2 =}\StringTok{ }\NormalTok{p2 }\OperatorTok{+}\StringTok{ }\KeywordTok{geom_vline}\NormalTok{(}\DataTypeTok{xintercept=}\KeywordTok{as.POSIXct}\NormalTok{(}\KeywordTok{as.Date}\NormalTok{((}\StringTok{"2016-06-22 UTC"}\NormalTok{))),}
                     \DataTypeTok{linetype=}\StringTok{"longdash"}\NormalTok{)}
\NormalTok{p2}
\end{Highlighting}
\end{Shaded}

\includegraphics[width=1\linewidth]{2018_02_07_IMF_FXCourse_files/figure-beamer/unnamed-chunk-9-1}

\end{frame}

\begin{frame}[fragile]{What the data tell us: domestic rate
differentials}

\begin{Shaded}
\begin{Highlighting}[]
\NormalTok{data}\OperatorTok{$}\NormalTok{spreadRd =}\StringTok{ }\NormalTok{(data}\OperatorTok{$}\NormalTok{ImpliedRd }\OperatorTok{-}\StringTok{ }\NormalTok{data}\OperatorTok{$}\NormalTok{Rd)}\OperatorTok{*}\DecValTok{100}
\NormalTok{p3 =}\StringTok{ }\KeywordTok{ggplot}\NormalTok{(data, }\KeywordTok{aes}\NormalTok{(Dates, spreadRd)) }\OperatorTok{+}\StringTok{ }\KeywordTok{geom_line}\NormalTok{(}\DataTypeTok{colour=}\StringTok{"purple"}\NormalTok{) }\OperatorTok{+}\StringTok{ }
\StringTok{        }\KeywordTok{geom_hline}\NormalTok{(}\DataTypeTok{yintercept=}\DecValTok{0}\NormalTok{)}
\NormalTok{p3 =}\StringTok{ }\NormalTok{p3 }\OperatorTok{+}\StringTok{ }\KeywordTok{geom_vline}\NormalTok{(}\DataTypeTok{xintercept=}\KeywordTok{as.POSIXct}\NormalTok{(}\KeywordTok{as.Date}\NormalTok{((}\StringTok{"2016-06-22 UTC"}\NormalTok{))),}
                     \DataTypeTok{linetype=}\StringTok{"longdash"}\NormalTok{)}
\NormalTok{p3}
\end{Highlighting}
\end{Shaded}

\includegraphics[width=1\linewidth]{2018_02_07_IMF_FXCourse_files/figure-beamer/unnamed-chunk-10-1}

\end{frame}

\begin{frame}[fragile]{What the data tell us: ATM volatility}

\begin{Shaded}
\begin{Highlighting}[]
\NormalTok{p4 =}\StringTok{ }\KeywordTok{ggplot}\NormalTok{(data, }\KeywordTok{aes}\NormalTok{(Dates, ATM)) }\OperatorTok{+}\StringTok{ }\KeywordTok{geom_line}\NormalTok{(}\DataTypeTok{colour=}\StringTok{"blue"}\NormalTok{) }\OperatorTok{+}\StringTok{ }
\StringTok{        }\KeywordTok{geom_hline}\NormalTok{(}\DataTypeTok{yintercept=}\DecValTok{0}\NormalTok{)}
\NormalTok{p4 =}\StringTok{ }\NormalTok{p4 }\OperatorTok{+}\StringTok{ }\KeywordTok{geom_vline}\NormalTok{(}\DataTypeTok{xintercept=}\KeywordTok{as.POSIXct}\NormalTok{(}\KeywordTok{as.Date}\NormalTok{((}\StringTok{"2016-06-22 UTC"}\NormalTok{))),}
                     \DataTypeTok{linetype=}\StringTok{"longdash"}\NormalTok{)}
\NormalTok{p4}
\end{Highlighting}
\end{Shaded}

\begin{figure}
\includegraphics[width=1\linewidth]{2018_02_07_IMF_FXCourse_files/figure-beamer/unnamed-chunk-11-1} \caption{ATM volatility}\label{fig:unnamed-chunk-11}
\end{figure}

\end{frame}

\begin{frame}{What the data tell us: risk reversals (1)}

\begin{figure}
\includegraphics[width=1\linewidth]{images/figRRDefinition} \caption{Risk reversal definition (Wikipedia)}\label{fig:unnamed-chunk-12}
\end{figure}

\end{frame}

\begin{frame}{What the data tell us: risk reversals (2)}

\begin{itemize}
\tightlist
\item
  Full understanding of the risk reversal requires knowing what
  \(\Delta\) is
\item
  But without knowing, we still use risk reversal quotes to assess the
  prices market participants place on potential exchange rate movements.
\end{itemize}

\[
RR_{25\Delta} = \sigma_{25\Delta C} - \sigma_{25\Delta P}
\]

\begin{itemize}
\tightlist
\item
  Pay \(\sigma_{25\Delta C}\) for owning the call
\item
  Offset cost somewhat by selling the put at \(\sigma_{25\Delta P}\)
\end{itemize}

\end{frame}

\begin{frame}{What the data tell us: risk reversals (3)}

\begin{itemize}
\tightlist
\item
  Risk reversal payoff, long a call and short a put, both of them OTM:
\end{itemize}

\begin{figure}
\includegraphics[width=0.7\linewidth]{images/fig25RRSimple} \caption{Risk reversal payoff at maturity}\label{fig:unnamed-chunk-13}
\end{figure}

\begin{itemize}
\tightlist
\item
  Explain why the price could be positive (or negative)?
\end{itemize}

\end{frame}

\begin{frame}[fragile]{What the data tell us: risk reversals (4)}

\begin{Shaded}
\begin{Highlighting}[]
\NormalTok{p5 =}\StringTok{ }\KeywordTok{ggplot}\NormalTok{(data, }\KeywordTok{aes}\NormalTok{(Dates, RR25D3M)) }\OperatorTok{+}\StringTok{ }\KeywordTok{geom_line}\NormalTok{(}\DataTypeTok{colour=}\StringTok{"red"}\NormalTok{) }\OperatorTok{+}\StringTok{ }
\StringTok{        }\KeywordTok{geom_hline}\NormalTok{(}\DataTypeTok{yintercept=}\DecValTok{0}\NormalTok{)}
\NormalTok{p5 =}\StringTok{ }\NormalTok{p5 }\OperatorTok{+}\StringTok{ }\KeywordTok{geom_vline}\NormalTok{(}\DataTypeTok{xintercept=}\KeywordTok{as.POSIXct}\NormalTok{(}\KeywordTok{as.Date}\NormalTok{((}\StringTok{"2016-06-22 UTC"}\NormalTok{))),}
                     \DataTypeTok{linetype=}\StringTok{"longdash"}\NormalTok{)}
\NormalTok{p5}
\end{Highlighting}
\end{Shaded}

\includegraphics[width=1\linewidth]{2018_02_07_IMF_FXCourse_files/figure-beamer/unnamed-chunk-14-1}

\end{frame}

\begin{frame}[fragile]{What the data tell us: risk reversals (5)}

\begin{Shaded}
\begin{Highlighting}[]
\NormalTok{p6 =}\StringTok{ }\KeywordTok{ggplot}\NormalTok{(data, }\KeywordTok{aes}\NormalTok{(Dates, RR10D3M)) }\OperatorTok{+}\StringTok{ }\KeywordTok{geom_line}\NormalTok{(}\DataTypeTok{colour=}\StringTok{"blue"}\NormalTok{) }\OperatorTok{+}\StringTok{ }
\StringTok{        }\KeywordTok{geom_hline}\NormalTok{(}\DataTypeTok{yintercept=}\DecValTok{0}\NormalTok{)}
\NormalTok{p6 =}\StringTok{ }\NormalTok{p6 }\OperatorTok{+}\StringTok{ }\KeywordTok{geom_vline}\NormalTok{(}\DataTypeTok{xintercept=}\KeywordTok{as.POSIXct}\NormalTok{(}\KeywordTok{as.Date}\NormalTok{((}\StringTok{"2016-06-22 UTC"}\NormalTok{))),}
                     \DataTypeTok{linetype=}\StringTok{"longdash"}\NormalTok{)}
\NormalTok{p6}
\end{Highlighting}
\end{Shaded}

\includegraphics[width=1\linewidth]{2018_02_07_IMF_FXCourse_files/figure-beamer/unnamed-chunk-15-1}

\end{frame}

\begin{frame}{What the data tells us: butterfly spreads (1)}

\begin{itemize}
\tightlist
\item
  Suppose we want to profit from large movements of the exchange rate
\item
  The strangle would make our day !!
\end{itemize}

\begin{center}\includegraphics[width=0.7\linewidth]{images/fig25BFSimple} \end{center}

\end{frame}

\begin{frame}{What the data tells us: butterfly spreads (2)}

\begin{itemize}
\tightlist
\item
  The cost of the strangle is:
\end{itemize}

\[
S_{25\Delta} = \sigma_{25\Delta C} + \sigma_{25\Delta P}
\]

\begin{itemize}
\tightlist
\item
  But markets don't quote strangles\\[2\baselineskip]
\item
  They quote \textbf{Butterfly Spreads}
\end{itemize}

\[
BF_{25\Delta} = \frac{\sigma_{25\Delta C} + \sigma_{25\Delta P}}{2}  - \sigma_{ATM}
\]

\end{frame}

\begin{frame}{What the data tells us: butterfly spreads (3)}

\begin{itemize}
\tightlist
\item
  Butterfly spreads convey more information
\end{itemize}

\begin{center}\includegraphics[width=1\linewidth]{images/figStrangleButterfly} \end{center}

\end{frame}

\begin{frame}[fragile]{What the data tells us: butterfly spreads (4)}

\begin{Shaded}
\begin{Highlighting}[]
\NormalTok{p7 =}\StringTok{ }\KeywordTok{ggplot}\NormalTok{(data, }\KeywordTok{aes}\NormalTok{(Dates, BF25D3M)) }\OperatorTok{+}\StringTok{ }\KeywordTok{geom_line}\NormalTok{(}\DataTypeTok{colour=}\StringTok{"red"}\NormalTok{) }\OperatorTok{+}\StringTok{ }
\StringTok{        }\KeywordTok{geom_hline}\NormalTok{(}\DataTypeTok{yintercept=}\DecValTok{0}\NormalTok{)}
\NormalTok{p7 =}\StringTok{ }\NormalTok{p7 }\OperatorTok{+}\StringTok{ }\KeywordTok{geom_vline}\NormalTok{(}\DataTypeTok{xintercept=}\KeywordTok{as.POSIXct}\NormalTok{(}\KeywordTok{as.Date}\NormalTok{((}\StringTok{"2016-06-22 UTC"}\NormalTok{))),}
                     \DataTypeTok{linetype=}\StringTok{"longdash"}\NormalTok{)}
\NormalTok{p7}
\end{Highlighting}
\end{Shaded}

\begin{center}\includegraphics[width=1\linewidth]{2018_02_07_IMF_FXCourse_files/figure-beamer/unnamed-chunk-18-1} \end{center}

\end{frame}

\begin{frame}[fragile]{What the data tells us: butterfly spreads (5)}

\begin{Shaded}
\begin{Highlighting}[]
\NormalTok{p8 =}\StringTok{ }\KeywordTok{ggplot}\NormalTok{(data, }\KeywordTok{aes}\NormalTok{(Dates, BF10D3M)) }\OperatorTok{+}\StringTok{ }\KeywordTok{geom_line}\NormalTok{(}\DataTypeTok{colour=}\StringTok{"blue"}\NormalTok{) }\OperatorTok{+}\StringTok{ }
\StringTok{        }\KeywordTok{geom_hline}\NormalTok{(}\DataTypeTok{yintercept=}\DecValTok{0}\NormalTok{)}
\NormalTok{p8 =}\StringTok{ }\NormalTok{p8 }\OperatorTok{+}\StringTok{ }\KeywordTok{geom_vline}\NormalTok{(}\DataTypeTok{xintercept=}\KeywordTok{as.POSIXct}\NormalTok{(}\KeywordTok{as.Date}\NormalTok{((}\StringTok{"2016-06-22 UTC"}\NormalTok{))),}
                     \DataTypeTok{linetype=}\StringTok{"longdash"}\NormalTok{)}
\NormalTok{p8}
\end{Highlighting}
\end{Shaded}

\begin{center}\includegraphics[width=1\linewidth]{2018_02_07_IMF_FXCourse_files/figure-beamer/unnamed-chunk-19-1} \end{center}

\end{frame}

\begin{frame}{}

\color{blue} \LARGE{Part C:}\\
\LARGE{A Little Bit of Option Pricing}

\end{frame}

\begin{frame}{What is \(\Delta\) (1)}

\begin{itemize}
\tightlist
\item
  The \(\Delta\) of a call
\end{itemize}

\[ \Delta_C = \frac{\partial C}{\partial S} \geq 0\]

\begin{itemize}
\tightlist
\item
  The \(\Delta\) of a put
\end{itemize}

\[ \Delta_P = \frac{\partial P}{\partial S} \leq 0\]

\end{frame}

\begin{frame}{What is \(\Delta\) (2)}

\begin{center}\includegraphics[width=0.8\linewidth]{images/fig25RR} \end{center}

\end{frame}

\begin{frame}{What is \(\Delta\) (3)}

\begin{center}\includegraphics[width=0.8\linewidth]{images/fig25Strangle} \end{center}

\end{frame}

\begin{frame}{What is \(\Delta\) (4)}

\begin{center}\includegraphics[width=1\linewidth]{images/fig1025RR} \end{center}

\end{frame}

\begin{frame}{Option prices = replication cost (1)}

Dealer costs when buying call option from client

\begin{itemize}
\tightlist
\item
  Funding cost: borrow \(C\) at \(R_d\): \(R_d C_t \delta t\)\\
\item
  Hedging cost (using delta-hedging)

  \begin{itemize}
  \tightlist
  \item
    Borrow and sell \(\Delta\) units of foreign currency
  \item
    Receive \(\Delta S\) and reinvest at \(R_d\)
  \item
    Pay accrued interest on borrowed amount of currency
  \item
    Net gain: \((R_d - R_f)\Delta S_t \delta t\)
  \end{itemize}
\end{itemize}

\end{frame}

\begin{frame}{Option price = replication cost (2)}

\begin{itemize}
\tightlist
\item
  Time decay of option: loses value as maturity approaches
  \[\frac{\partial C_t}{\partial t} \delta t\]\\
  \hspace*{0.333em}
\item
  Convexity gains since options are non-linear
  \[ \frac{\partial^2 C_t}{\partial S^2}(\delta S_t)^2\]
\end{itemize}

\end{frame}

\begin{frame}{Option price = replication cost (3)}

\begin{itemize}
\tightlist
\item
  Gains must offset costs, yielding the pricing partial differential
  equation:
\end{itemize}

\[
\frac{\partial C_t}{\partial t} \delta t + (R_d - R_f) \Delta S_t  \delta t + \frac{\partial^2 C}{\partial S_t^2} (\delta S_t)^2 = R_d C_t \delta t
\]

\end{frame}

\begin{frame}{Garman-Kohlhagen formula (1)}

Assume FX follows a geometric brownian motion, the call option price is:

\[
\boxed{
C(K,S_t,R_d,R_f,T,\sigma) = S_t \exp(-R_f\times(T-t))N(d_1) - K\exp(-R_d \times (T-t))N(d_2)
}
\]

and the put option price is:

\[
\boxed{
P(K,S_t,R_d,R_f,T,\sigma) = K\exp(-R_d \times (T-t))N(-d_2) - S_t \exp(-R_f\times(T-t))N(-d_1) 
}
\]

\end{frame}

\begin{frame}{Garman-Kohlhagen formula (2)}

\begin{itemize}
\tightlist
\item
  \(K\) is the strike price,
\item
  \(S_t\) is the current spot exchange rate,
\item
  \(R_d\) is the domestic interest rate,
\item
  \(R_f\) is the foreign interest rate,
\item
  \(T-t\) is the remaining life of an option maturing at time \(T\),
\item
  \(\sigma\) is the implied volatility of the exchange rate used to
  price the option,
\item
  \[
  \begin{aligned}
  d_1 &= \frac{\ln(S_t/K)+(R_d-R_f+\sigma^2/2)(T-t)}{\sigma \times (T-t)}\\
  d_2 &= d_1 - \sigma \times (T-t)
  \end{aligned}
  \]
\end{itemize}

\end{frame}

\begin{frame}{Implied volatility \(\sigma\) (1)}

\begin{itemize}
\tightlist
\item
  All GK formula inputs are observable except implied volatility
  \(\sigma\)\\[2\baselineskip]
\item
  Option prices are quoted as \textbf{vols},
  i.e.~volatility\\[2\baselineskip]
\item
  To obtain price

  \begin{itemize}
  \tightlist
  \item
    Take quoted vol
  \item
    Use reference values for all other variables
  \item
    Plug into GK formula
  \end{itemize}
\end{itemize}

\end{frame}

\begin{frame}{Implied volatility \(\sigma\) (2)}

Implied volatility

\begin{itemize}
\tightlist
\item
  is different from historical or realized volatility\\[2\baselineskip]
\item
  is not a forecast of future volatility\\[2\baselineskip]
\item
  yields an GK option premium reflecting

  \begin{itemize}
  \tightlist
  \item
    profit margin
  \item
    hedging costs
  \item
    demand and supply in the FX market
  \end{itemize}
\end{itemize}

\end{frame}

\begin{frame}{}

\color{blue} \LARGE{Part D:}\\
\LARGE{The Volatility Smile}

\end{frame}

\begin{frame}{Finding vols for different \(\Delta\)s}

\begin{itemize}
\item
  For a given \(\Delta\) we have prices for:

  \begin{itemize}
  \tightlist
  \item
    Risk reversal (RR)
  \item
    Butterfly spread (BF)
  \end{itemize}
\item
  The ATM vol, \(\sigma_{ATM}\) is also given
\item
  From the definitions of the RR and the BF
\end{itemize}

\[
\begin{aligned}
  \sigma_{25\Delta C} &= \sigma_{ATM} + BF_{25\Delta} + \frac{1}{2} RR_{25\Delta}\\
  \sigma_{25\Delta P} &= \sigma_{ATM} + BF_{25\Delta} - \frac{1}{2} RR_{25\Delta}\\
  \sigma_{75\Delta C} &= \sigma_{25\Delta P}\enspace \text{by put-call parity}
\end{aligned}
\]

\begin{itemize}
\tightlist
\item
  Plotting \(\sigma\) against \(\Delta\) yields the volatility smile
\end{itemize}

\end{frame}

\begin{frame}{The level of the smile: \(\sigma_{ATM}\)}

\includegraphics[width=1\linewidth]{images/figRRLevel}

\end{frame}

\begin{frame}{The slope of the smile: the risk reversal}

\includegraphics[width=1\linewidth]{images/figRRSlope}

\end{frame}

\begin{frame}{The curvature of the smile: the butterfly spread}

\includegraphics[width=1\linewidth]{images/figRRCurvature}

\end{frame}

\begin{frame}{}

\color{blue} \LARGE{Part E:}\\
\LARGE{Constructing the Volatility Smile} ~ \Large{A Brexit case study}

\end{frame}

\begin{frame}{Selecting the dates}

We are interested in the behavior of the GBPUSD for three dates:

\begin{itemize}
\tightlist
\item
  January 8, 2016 (Pre-Brexit)
\item
  June 24, 2016 (Brexit)
\item
  December 30, 2016 (Post-Brexit)
\end{itemize}

\end{frame}

\begin{frame}[fragile]{Getting the data (1)}

\begin{Shaded}
\begin{Highlighting}[]
\KeywordTok{rm}\NormalTok{(}\DataTypeTok{list=}\KeywordTok{ls}\NormalTok{())                                     }\CommentTok{# Clean up memory}
\NormalTok{filename =}\StringTok{ "2018_IET_Options_data.csv"}            \CommentTok{# Name of CSV data file}
\NormalTok{data =}\StringTok{ }\KeywordTok{read.csv}\NormalTok{(filename, }\DataTypeTok{header=}\OtherTok{TRUE}\NormalTok{)            }\CommentTok{# Load datafile}
\NormalTok{data}\OperatorTok{$}\NormalTok{Dates =}\StringTok{ }\KeywordTok{mdy_hm}\NormalTok{(}\KeywordTok{as.character}\NormalTok{(data}\OperatorTok{$}\NormalTok{Dates))     }\CommentTok{# convert dates to Date class}
\KeywordTok{rownames}\NormalTok{(data)=}\OtherTok{NULL}                               \CommentTok{# remove row names}

\CommentTok{# Specify dates for analysis}

\NormalTok{date01 =}\StringTok{ }\KeywordTok{as.Date}\NormalTok{(}\StringTok{"2016-01-08 UTC"}\NormalTok{)                 }
\NormalTok{date02 =}\StringTok{ }\KeywordTok{as.Date}\NormalTok{(}\StringTok{"2016-06-24 UTC"}\NormalTok{)}
\NormalTok{date03 =}\StringTok{ }\KeywordTok{as.Date}\NormalTok{(}\StringTok{"2016-12-30 UTC"}\NormalTok{)}
\end{Highlighting}
\end{Shaded}

\end{frame}

\begin{frame}[fragile]{Getting the data (2)}

\begin{Shaded}
\begin{Highlighting}[]
\CommentTok{# Create data frame this.data}

\NormalTok{this.data =}\StringTok{ }\KeywordTok{rbind}\NormalTok{(}
\NormalTok{  data[}\KeywordTok{which}\NormalTok{(data}\OperatorTok{$}\NormalTok{Dates}\OperatorTok{==}\NormalTok{date01),],}
\NormalTok{  data[}\KeywordTok{which}\NormalTok{(data}\OperatorTok{$}\NormalTok{Dates}\OperatorTok{==}\NormalTok{date02),],}
\NormalTok{  data[}\KeywordTok{which}\NormalTok{(data}\OperatorTok{$}\NormalTok{Dates}\OperatorTok{==}\NormalTok{date03),])}

\CommentTok{# Delete row names and change the names of the columns}

\KeywordTok{rownames}\NormalTok{(this.data) =}\StringTok{ }\OtherTok{NULL}
\KeywordTok{colnames}\NormalTok{(this.data) =}\StringTok{ }\KeywordTok{c}\NormalTok{(}\StringTok{"Date"}\NormalTok{,}\StringTok{"spot"}\NormalTok{,}\StringTok{"forward"}\NormalTok{,}\StringTok{"atm"}\NormalTok{, }\StringTok{"rr25"}\NormalTok{,}\StringTok{"bf25"}\NormalTok{,}
                        \StringTok{"rr10"}\NormalTok{,}\StringTok{"bf10"}\NormalTok{,}\StringTok{"rf"}\NormalTok{,}\StringTok{"rd"}\NormalTok{,}\StringTok{"imp_rd"}\NormalTok{)}
\end{Highlighting}
\end{Shaded}

\end{frame}

\begin{frame}{Get additional vols}

In addition to the 25\(\Delta\) and 75\(\Delta\) vols, obtain the
10\(\Delta\) and 90\(\Delta\) vols:

\[
\begin{aligned}
  \sigma_{10\Delta C} &= \sigma_{ATM} + BF_{10\Delta} + \frac{1}{2} RR_{10\Delta}\\
  \sigma_{10\Delta P} &= \sigma_{ATM} + BF_{10\Delta} - \frac{1}{2} RR_{10\Delta}\\
  \sigma_{90\Delta C} &= \sigma_{10\Delta P}
\end{aligned}
\]

\end{frame}

\begin{frame}[fragile]{Calculating the implied vols (1)}

\begin{Shaded}
\begin{Highlighting}[]
\CommentTok{# Vols are in percent, expressed them as simple numbers}
\NormalTok{this.data}\OperatorTok{$}\NormalTok{atm  =}\StringTok{ }\NormalTok{this.data}\OperatorTok{$}\NormalTok{atm}\OperatorTok{/}\DecValTok{100}
\NormalTok{this.data}\OperatorTok{$}\NormalTok{rr25 =}\StringTok{ }\NormalTok{this.data}\OperatorTok{$}\NormalTok{rr25}\OperatorTok{/}\DecValTok{100}
\NormalTok{this.data}\OperatorTok{$}\NormalTok{bf25 =}\StringTok{ }\NormalTok{this.data}\OperatorTok{$}\NormalTok{bf25}\OperatorTok{/}\DecValTok{100}
\NormalTok{this.data}\OperatorTok{$}\NormalTok{rr10 =}\StringTok{ }\NormalTok{this.data}\OperatorTok{$}\NormalTok{rr10}\OperatorTok{/}\DecValTok{100}
\NormalTok{this.data}\OperatorTok{$}\NormalTok{bf10 =}\StringTok{ }\NormalTok{this.data}\OperatorTok{$}\NormalTok{bf10}\OperatorTok{/}\DecValTok{100}
\NormalTok{this.data}\OperatorTok{$}\NormalTok{rf   =}\StringTok{ }\NormalTok{this.data}\OperatorTok{$}\NormalTok{rf}\OperatorTok{/}\DecValTok{100}
\NormalTok{this.data}\OperatorTok{$}\NormalTok{rd   =}\StringTok{ }\NormalTok{this.data}\OperatorTok{$}\NormalTok{rd}\OperatorTok{/}\DecValTok{100}
\NormalTok{this.data}\OperatorTok{$}\NormalTok{imp_rd=this.data}\OperatorTok{$}\NormalTok{imp_rd}\OperatorTok{/}\DecValTok{100}

\CommentTok{# We will use this.data repeatedly}
\CommentTok{# Attach it to access its elements}

\KeywordTok{attach}\NormalTok{(this.data)     }
\end{Highlighting}
\end{Shaded}

\end{frame}

\begin{frame}[fragile]{Calculating the implied vols (2)}

\begin{Shaded}
\begin{Highlighting}[]
\CommentTok{# Recover vols for different deltas and put them in the data frame}

\NormalTok{this.data}\OperatorTok{$}\NormalTok{sigma10c =}\StringTok{ }\NormalTok{atm }\OperatorTok{+}\StringTok{ }\NormalTok{bf10 }\OperatorTok{+}\StringTok{ }\FloatTok{0.5}\OperatorTok{*}\NormalTok{rr10}
\NormalTok{this.data}\OperatorTok{$}\NormalTok{sigma25c =}\StringTok{ }\NormalTok{atm }\OperatorTok{+}\StringTok{ }\NormalTok{bf25 }\OperatorTok{+}\StringTok{ }\FloatTok{0.5}\OperatorTok{*}\NormalTok{rr25}
\NormalTok{this.data}\OperatorTok{$}\NormalTok{sigma75c =}\StringTok{ }\NormalTok{this.data}\OperatorTok{$}\NormalTok{sigma25c }\OperatorTok{-}\StringTok{ }\NormalTok{rr25}
\NormalTok{this.data}\OperatorTok{$}\NormalTok{sigma90c =}\StringTok{ }\NormalTok{this.data}\OperatorTok{$}\NormalTok{sigma10c }\OperatorTok{-}\StringTok{ }\NormalTok{rr10}
\NormalTok{this.data}\OperatorTok{$}\NormalTok{sigmaatm =}\StringTok{ }\NormalTok{atm}

\NormalTok{Tenor =}\StringTok{ }\DecValTok{3}\OperatorTok{/}\DecValTok{12}   \CommentTok{# Maturity of options, 3 months, in years}
\end{Highlighting}
\end{Shaded}

\end{frame}

\begin{frame}{What is the proper \(ATM\) strike?}

\begin{itemize}
\tightlist
\item
  Retail products \[
  \begin{aligned}
  K_{ATM} &= S\\
  \Delta_{ATM} &= N \left(\frac{\log(F/S) + \frac{1}{2} \sigma_{ATM}^2 T}{\sigma_{ATM} \sqrt T} \right)
  \end{aligned}
  \]
\item
  EM currencies, maturities more than one year \[
  \begin{aligned}
  K_{ATM} &= F\\
  \Delta_{ATM} &= \exp(-Rf \times T)N(\frac{1}{2}\sigma_{ATM} \sqrt T) 
  \end{aligned}
  \]
\item
  Major currencies, maturities of one year or less \[
  \begin{aligned}
   K_{ATM} &= F \times \exp\left( 0.5 \sigma^2_{ATM}\times T \right)\\
   \Delta_{ATM} &= 0.5\times exp(-Rf \times T) \simeq 0.5 
  \end{aligned}
  \]
\end{itemize}

\end{frame}

\begin{frame}[fragile]{Calculate the \(\Delta_{ATM}\)}

\begin{Shaded}
\begin{Highlighting}[]
\CommentTok{# Calculate the strike of the ATM option}
\NormalTok{K_atm =}\StringTok{ }\NormalTok{forward}\OperatorTok{*}\KeywordTok{exp}\NormalTok{((}\FloatTok{0.5}\OperatorTok{*}\NormalTok{(this.data}\OperatorTok{$}\NormalTok{sigmaatm)}\OperatorTok{^}\DecValTok{2}\NormalTok{)}\OperatorTok{*}\NormalTok{Tenor)}
\NormalTok{deltaATM =}\StringTok{ }\FloatTok{0.5}\OperatorTok{*}\KeywordTok{exp}\NormalTok{(}\OperatorTok{-}\NormalTok{rf}\OperatorTok{*}\NormalTok{Tenor)}
\NormalTok{deltaATM}
\end{Highlighting}
\end{Shaded}

\begin{verbatim}
## [1] 0.4989636 0.4993504 0.4994441
\end{verbatim}

\end{frame}

\begin{frame}[fragile]{Rough volatility smile (1)}

\begin{Shaded}
\begin{Highlighting}[]
\CommentTok{# Select only the vols for each delta}
\NormalTok{list_variables =}\StringTok{ }\KeywordTok{c}\NormalTok{(}\StringTok{"sigma10c"}\NormalTok{, }\StringTok{"sigma25c"}\NormalTok{, }\StringTok{"sigmaatm"}\NormalTok{, }
                   \StringTok{"sigma75c"}\NormalTok{, }\StringTok{"sigma90c"}\NormalTok{)}

\CommentTok{# Read the data as a matrix}
\NormalTok{vol_data =}\StringTok{ }\KeywordTok{t}\NormalTok{(}\KeywordTok{as.matrix}\NormalTok{(}\KeywordTok{subset}\NormalTok{(this.data, }\DataTypeTok{select=}\NormalTok{list_variables)))}

\CommentTok{# Group the deltas in a vector, to be used in the x-axis}
\NormalTok{delta_vector =}\StringTok{ }\KeywordTok{c}\NormalTok{(}\FloatTok{0.10}\NormalTok{, }\FloatTok{0.25}\NormalTok{, }\FloatTok{0.5}\NormalTok{, }\FloatTok{0.75}\NormalTok{, }\FloatTok{0.9}\NormalTok{)}

\CommentTok{# Create the data frame for the chart}
\NormalTok{vol.smile  =}\StringTok{ }\KeywordTok{data.frame}\NormalTok{(delta_vector, vol_data)   }
\KeywordTok{rownames}\NormalTok{(vol.smile) =}\StringTok{ }\OtherTok{NULL}
\KeywordTok{colnames}\NormalTok{(vol.smile) =}\StringTok{ }\KeywordTok{c}\NormalTok{(}\StringTok{"Delta"}\NormalTok{,}\StringTok{"PreBrexit"}\NormalTok{,}\StringTok{"Brexit"}\NormalTok{,}\StringTok{"PostBrexit"}\NormalTok{)}
\end{Highlighting}
\end{Shaded}

\end{frame}

\begin{frame}[fragile]{Rough volatility smile (2)}

\begin{Shaded}
\begin{Highlighting}[]
\KeywordTok{library}\NormalTok{(reshape2)}
\NormalTok{vol.data =}\StringTok{ }\KeywordTok{melt}\NormalTok{(vol.smile, }\DataTypeTok{id=}\StringTok{"Delta"}\NormalTok{)}
\KeywordTok{ggplot}\NormalTok{(}\DataTypeTok{data=}\NormalTok{vol.data, }\KeywordTok{aes}\NormalTok{(}\DataTypeTok{x=}\NormalTok{Delta, }\DataTypeTok{y=}\NormalTok{value, }\DataTypeTok{shape=}\NormalTok{variable)) }\OperatorTok{+}
\StringTok{   }\KeywordTok{geom_point}\NormalTok{(}\KeywordTok{aes}\NormalTok{(}\DataTypeTok{colour=}\NormalTok{variable), }\DataTypeTok{size=}\DecValTok{4}\NormalTok{) }\OperatorTok{+}
\StringTok{   }\KeywordTok{labs}\NormalTok{(}\DataTypeTok{y=}\StringTok{"vol"}\NormalTok{)}
\end{Highlighting}
\end{Shaded}

\end{frame}

\begin{frame}{Rough volatility smile (3)}

\includegraphics[width=1\linewidth]{2018_02_07_IMF_FXCourse_files/figure-beamer/unnamed-chunk-33-1}

\end{frame}

\begin{frame}[fragile]{A more refined volatility smile (1)}

\begin{Shaded}
\begin{Highlighting}[]
\CommentTok{# Fit second degree polynomial to smile}
\NormalTok{fit.Vol =}\StringTok{ }\ControlFlowTok{function}\NormalTok{(data.vol, data.delta,delta.range)}
\NormalTok{\{}
\NormalTok{  poly.fit =}\StringTok{ }\KeywordTok{lm}\NormalTok{(data.vol }\OperatorTok{~}\StringTok{ }\KeywordTok{poly}\NormalTok{(data.delta, }\DecValTok{2}\NormalTok{, }\DataTypeTok{raw=}\OtherTok{TRUE}\NormalTok{))  }

  \CommentTok{# Use fitted polynomial to interpolate Delta-Vol Curve}
\NormalTok{  delta.square =}\StringTok{ }\NormalTok{delta.range}\OperatorTok{*}\NormalTok{delta.range}
\NormalTok{  delta.interc =}\StringTok{ }\KeywordTok{rep}\NormalTok{(}\DecValTok{1}\NormalTok{,}\KeywordTok{length}\NormalTok{(delta.range))}
  
\NormalTok{  X =}\StringTok{ }\KeywordTok{cbind}\NormalTok{(delta.interc, delta.range, delta.square)}
\NormalTok{  iVolInterpol =}\StringTok{ }\KeywordTok{t}\NormalTok{(}\KeywordTok{t}\NormalTok{(X)}\OperatorTok{*}\NormalTok{poly.fit}\OperatorTok{$}\NormalTok{coefficients)}
\NormalTok{  iVolInterpol =}\StringTok{ }\KeywordTok{rowSums}\NormalTok{(iVolInterpol)}

  \KeywordTok{return}\NormalTok{(iVolInterpol)}
\NormalTok{\}}
\end{Highlighting}
\end{Shaded}

\end{frame}

\begin{frame}[fragile]{A more refined volatility smile (2)}

\begin{Shaded}
\begin{Highlighting}[]
\CommentTok{# Get the data points from the vol.smile data frame}
\NormalTok{data.delta =}\StringTok{ }\NormalTok{vol.smile}\OperatorTok{$}\NormalTok{Delta}
\NormalTok{data.vol  =}\StringTok{ }\NormalTok{vol.smile}\OperatorTok{$}\NormalTok{PreBrexit}

\CommentTok{# Interpolation and extrapolation range}
\NormalTok{delta.range =}\StringTok{ }\KeywordTok{seq}\NormalTok{(}\DataTypeTok{from=}\FloatTok{0.01}\NormalTok{, }\DataTypeTok{to =} \FloatTok{0.99}\NormalTok{, }\DataTypeTok{by=}\FloatTok{0.005}\NormalTok{)}

\CommentTok{# Obtain the volatility smiles}
\NormalTok{smile.preBrexit =}\KeywordTok{fit.Vol}\NormalTok{(vol.smile}\OperatorTok{$}\NormalTok{PreBrexit, data.delta, delta.range)}
\NormalTok{smile.Brexit    =}\KeywordTok{fit.Vol}\NormalTok{(vol.smile}\OperatorTok{$}\NormalTok{Brexit, data.delta, delta.range)}
\NormalTok{smile.postBrexit=}\KeywordTok{fit.Vol}\NormalTok{(vol.smile}\OperatorTok{$}\NormalTok{PostBrexit, data.delta, delta.range)}
\end{Highlighting}
\end{Shaded}

\end{frame}

\begin{frame}[fragile]{A more refined volatility smile (3)}

\begin{Shaded}
\begin{Highlighting}[]
\CommentTok{# Group the extended volatility smiles in a data frame}
\NormalTok{smile.df =}\StringTok{ }\KeywordTok{data.frame}\NormalTok{(delta.range, smile.preBrexit, }
\NormalTok{                      smile.Brexit, smile.postBrexit)}
\KeywordTok{rownames}\NormalTok{(smile.df) =}\StringTok{ }\OtherTok{NULL}
\KeywordTok{colnames}\NormalTok{(smile.df) =}\StringTok{ }\KeywordTok{c}\NormalTok{(}\StringTok{"Delta"}\NormalTok{,}\StringTok{"preBrexit"}\NormalTok{,}\StringTok{"Brexit"}\NormalTok{, }\StringTok{"postBrexit"}\NormalTok{)}

\CommentTok{# Create the chart}
\KeywordTok{library}\NormalTok{(reshape2)}
\NormalTok{smile.data =}\StringTok{ }\KeywordTok{melt}\NormalTok{(smile.df, }\DataTypeTok{id=}\StringTok{"Delta"}\NormalTok{)}
\KeywordTok{ggplot}\NormalTok{(}\DataTypeTok{data =}\NormalTok{ smile.data, }\KeywordTok{aes}\NormalTok{(}\DataTypeTok{x=}\NormalTok{Delta, }\DataTypeTok{y=}\NormalTok{value, }\DataTypeTok{colour=}\NormalTok{variable)) }\OperatorTok{+}
\StringTok{  }\KeywordTok{geom_line}\NormalTok{(}\DataTypeTok{size=}\DecValTok{1}\NormalTok{) }\OperatorTok{+}\StringTok{ }\KeywordTok{labs}\NormalTok{(}\DataTypeTok{y=}\StringTok{"vol"}\NormalTok{)}
\end{Highlighting}
\end{Shaded}

\end{frame}

\begin{frame}{A more refined volatility smile (4)}

\includegraphics[width=1\linewidth]{2018_02_07_IMF_FXCourse_files/figure-beamer/unnamed-chunk-37-1}

\end{frame}

\begin{frame}{}

\color{blue} \LARGE{Part F:}\\
\LARGE{Extracting Risk-Neutral Densities}

\end{frame}

\begin{frame}{Market views are not forecasts}

Market views

\begin{itemize}
\tightlist
\item
  They are not forecast of the real-world FX distribution
\item
  Market prices weigh risk aversion of

  \begin{itemize}
  \tightlist
  \item
    hedgers
  \item
    speculators
  \item
    market makers
  \end{itemize}
\item
  ``Technical'' factors may affect weight of events
\item
  Risk-neutral, weighting expectations and risk aversion

  \begin{itemize}
  \tightlist
  \item
    Not easy to disentangle them
  \end{itemize}
\end{itemize}

\end{frame}

\begin{frame}{Extraction methods}

Methods fall under one of three categories:

\begin{itemize}
\tightlist
\item
  Parametric methods
\item
  Semiparametric methods
\item
  Non-parametric methods
\end{itemize}

\end{frame}

\begin{frame}{Parametric methods}

\begin{itemize}
\tightlist
\item
  Specify a parametric distribution function \(F\) for the exchange rate
\item
  For the distribution parameters \(\Lambda\), the price of a call
  option is
\end{itemize}

\[
C^F(K) = \exp(-R_d\times T) \int_{K}^{\infty} \left( S_T - K\right) dF(S_T; \Lambda), 
\]

\begin{itemize}
\tightlist
\item
  The parameters that fit the market-implied distribution better solve
\end{itemize}

\[
\arg\min_{\Lambda} \sum_{j=1}^N |C^F(K_j) - C^{\text{Market}}(K_j)|^2
\]

\end{frame}

\begin{frame}{Semi-parametric methods}

\begin{itemize}
\tightlist
\item
  Start with a particular density
\item
  Add expansion terms to distribution
\item
  Each term helps to approximate market-implied distribution
\end{itemize}

\end{frame}

\begin{frame}{Non-parametric methods}

\begin{itemize}
\tightlist
\item
  Plot value of calls against different strike prices, i.e.
  \textbf{market call function}
\item
  Breeden and Litzenberger: risk-neutral distribution proportional to
  second derivative of the call function: \[
  \frac{\partial^2 C}{\partial K^2} \bigg\rvert_{K=S} = \exp(-R_d \times T)q(S).
  \]
\item
  Use numerical methods to calculate derivative and find risk neutral
  distribution
\end{itemize}

\end{frame}

\begin{frame}{Numerical estimation exercise}

\begin{itemize}
\tightlist
\item
  Risk neutral density extraction

  \begin{itemize}
  \tightlist
  \item
    Generalized beta density (parametric)
  \item
    Edgeworth expansion (semi-parametric)
  \item
    Shimko (non-parametric)
  \end{itemize}
\item
  All methods require constructing call and/or put functions

  \begin{itemize}
  \tightlist
  \item
    Option premia vs.~strike price
  \end{itemize}
\end{itemize}

\end{frame}

\begin{frame}{}

\color{blue} \LARGE{Part F:}\\
\LARGE{Extracting Risk-Neutral Densities}\\
\Large{Constructing option premium functions}

\end{frame}

\begin{frame}[fragile]{Constructing the market call function (1)}

\begin{itemize}
\tightlist
\item
  We need to obtain the market call function
\item
  In other words, move from \(\Delta\) - \(\sigma\) space to
  \emph{Option premium} - \emph{Strike} space
\item
  Two useful functions in \texttt{auxFunctions.R}

  \begin{itemize}
  \tightlist
  \item
    \texttt{get.strike()}
  \item
    \texttt{GKoption.premium()}
  \end{itemize}
\end{itemize}

\end{frame}

\begin{frame}[fragile]{Constructing the market call function (2)}

\begin{Shaded}
\begin{Highlighting}[]
\NormalTok{get.strike =}\StringTok{ }\ControlFlowTok{function}\NormalTok{(vol,delta,S0,fwd,rf,Tenor)}
\NormalTok{\{}
\NormalTok{  aux =}\StringTok{ }\KeywordTok{qnorm}\NormalTok{(delta}\OperatorTok{*}\KeywordTok{exp}\NormalTok{(rf}\OperatorTok{*}\NormalTok{Tenor))}
\NormalTok{  aux =}\StringTok{ }\NormalTok{aux}\OperatorTok{*}\NormalTok{vol}\OperatorTok{*}\KeywordTok{sqrt}\NormalTok{(Tenor)}
\NormalTok{  aux =}\StringTok{ }\NormalTok{aux }\OperatorTok{-}\StringTok{ }\FloatTok{0.5}\OperatorTok{*}\NormalTok{vol}\OperatorTok{*}\NormalTok{vol}\OperatorTok{*}\NormalTok{Tenor}
\NormalTok{  K =}\StringTok{ }\KeywordTok{exp}\NormalTok{(}\OperatorTok{-}\NormalTok{aux)}\OperatorTok{*}\NormalTok{fwd}
  \KeywordTok{return}\NormalTok{(K)}
\NormalTok{\}}
\end{Highlighting}
\end{Shaded}

\end{frame}

\begin{frame}[fragile]{Constructing the market call function (3)}

\begin{Shaded}
\begin{Highlighting}[]
\NormalTok{GKoption.premium =}\StringTok{ }\ControlFlowTok{function}\NormalTok{(K,sigma,S,Tenor,fwd,rf,option_type)}
\NormalTok{\{}
  \ControlFlowTok{if}\NormalTok{ (option_type }\OperatorTok{==}\StringTok{"c"}\NormalTok{) \{w=}\DecValTok{1}\NormalTok{\}}
  \ControlFlowTok{if}\NormalTok{ (option_type }\OperatorTok{==}\StringTok{"p"}\NormalTok{) \{w=}\OperatorTok{-}\DecValTok{1}\NormalTok{\}}
\NormalTok{  d1 =}\StringTok{ }\KeywordTok{log}\NormalTok{(fwd}\OperatorTok{/}\NormalTok{K)}\OperatorTok{+}\FloatTok{0.5}\OperatorTok{*}\NormalTok{sigma}\OperatorTok{*}\NormalTok{sigma}\OperatorTok{*}\NormalTok{Tenor}
\NormalTok{  d1 =}\StringTok{ }\NormalTok{d1}\OperatorTok{/}\NormalTok{(sigma}\OperatorTok{*}\KeywordTok{sqrt}\NormalTok{(Tenor))}
\NormalTok{  d2 =}\StringTok{ }\NormalTok{d1 }\OperatorTok{-}\StringTok{ }\NormalTok{sigma}\OperatorTok{*}\KeywordTok{sqrt}\NormalTok{(Tenor)}
\NormalTok{  rd =}\StringTok{ }\KeywordTok{log}\NormalTok{(fwd}\OperatorTok{/}\NormalTok{S)}\OperatorTok{/}\NormalTok{Tenor }\OperatorTok{+}\StringTok{ }\NormalTok{rf}
\NormalTok{  premium =}\StringTok{ }\KeywordTok{exp}\NormalTok{(}\OperatorTok{-}\NormalTok{rd}\OperatorTok{*}\NormalTok{Tenor)}\OperatorTok{*}\NormalTok{(w}\OperatorTok{*}\NormalTok{fwd}\OperatorTok{*}\KeywordTok{pnorm}\NormalTok{(w}\OperatorTok{*}\NormalTok{d1) }\OperatorTok{-}\StringTok{ }\NormalTok{w}\OperatorTok{*}\NormalTok{K}\OperatorTok{*}\KeywordTok{pnorm}\NormalTok{(w}\OperatorTok{*}\NormalTok{d2))}
  \KeywordTok{return}\NormalTok{(premium)}
\NormalTok{\}}
\end{Highlighting}
\end{Shaded}

\end{frame}

\begin{frame}[fragile]{Constructing the market call function (3)}

Strike ranges for the selected dates:

\begin{Shaded}
\begin{Highlighting}[]
\NormalTok{K_preBrexit =}\StringTok{ }\KeywordTok{mapply}\NormalTok{(get.strike,smile.df}\OperatorTok{$}\NormalTok{preBrexit, smile.df}\OperatorTok{$}\NormalTok{Delta, }
                     \DataTypeTok{S0=}\NormalTok{spot[}\DecValTok{1}\NormalTok{], }\DataTypeTok{fwd=}\NormalTok{forward[}\DecValTok{1}\NormalTok{], }\DataTypeTok{rf=}\NormalTok{rf[}\DecValTok{1}\NormalTok{], }\DataTypeTok{Tenor=}\FloatTok{0.25}\NormalTok{)}
\NormalTok{K_Brexit    =}\StringTok{ }\KeywordTok{mapply}\NormalTok{(get.strike,smile.df}\OperatorTok{$}\NormalTok{Brexit, smile.df}\OperatorTok{$}\NormalTok{Delta, }
                     \DataTypeTok{S0=}\NormalTok{spot[}\DecValTok{2}\NormalTok{], }\DataTypeTok{fwd=}\NormalTok{forward[}\DecValTok{2}\NormalTok{], }\DataTypeTok{rf=}\NormalTok{rf[}\DecValTok{2}\NormalTok{], }\DataTypeTok{Tenor=}\FloatTok{0.25}\NormalTok{)}
\NormalTok{K_postBrexit=}\StringTok{ }\KeywordTok{mapply}\NormalTok{(get.strike,smile.df}\OperatorTok{$}\NormalTok{postBrexit, smile.df}\OperatorTok{$}\NormalTok{Delta, }
                     \DataTypeTok{S0=}\NormalTok{spot[}\DecValTok{3}\NormalTok{], }\DataTypeTok{fwd=}\NormalTok{forward[}\DecValTok{3}\NormalTok{], }\DataTypeTok{rf=}\NormalTok{rf[}\DecValTok{3}\NormalTok{], }\DataTypeTok{Tenor=}\FloatTok{0.25}\NormalTok{)}
\end{Highlighting}
\end{Shaded}

\end{frame}

\begin{frame}[fragile]{Constructing the market call function (4)}

Generate the call premia:

\begin{Shaded}
\begin{Highlighting}[]
\NormalTok{Call_preBrexit =}\StringTok{ }\KeywordTok{mapply}\NormalTok{(GKoption.premium, K_preBrexit, }
\NormalTok{                        smile.df}\OperatorTok{$}\NormalTok{preBrexit, }\DataTypeTok{S=}\NormalTok{spot[}\DecValTok{1}\NormalTok{],}\DataTypeTok{Tenor=}\NormalTok{Tenor,}
                        \DataTypeTok{fwd=}\NormalTok{forward[}\DecValTok{1}\NormalTok{],}\DataTypeTok{rf=}\NormalTok{rf[}\DecValTok{1}\NormalTok{], }\DataTypeTok{option_type=}\StringTok{"c"}\NormalTok{)}

\NormalTok{Call_Brexit =}\StringTok{ }\KeywordTok{mapply}\NormalTok{(GKoption.premium, K_Brexit, }
\NormalTok{                     smile.df}\OperatorTok{$}\NormalTok{Brexit, }\DataTypeTok{S=}\NormalTok{spot[}\DecValTok{2}\NormalTok{],}\DataTypeTok{Tenor=}\NormalTok{Tenor,}
                     \DataTypeTok{fwd=}\NormalTok{forward[}\DecValTok{2}\NormalTok{],}\DataTypeTok{rf=}\NormalTok{rf[}\DecValTok{2}\NormalTok{],}\DataTypeTok{option_type=}\StringTok{"c"}\NormalTok{)}

\NormalTok{Call_postBrexit =}\StringTok{ }\KeywordTok{mapply}\NormalTok{(GKoption.premium, K_postBrexit, }
\NormalTok{                         smile.df}\OperatorTok{$}\NormalTok{postBrexit,}\DataTypeTok{S=}\NormalTok{spot[}\DecValTok{3}\NormalTok{],}\DataTypeTok{Tenor=}\NormalTok{Tenor,}
                         \DataTypeTok{fwd=}\NormalTok{forward[}\DecValTok{3}\NormalTok{],}\DataTypeTok{rf=}\NormalTok{rf[}\DecValTok{3}\NormalTok{],}\DataTypeTok{option_type=}\StringTok{"c"}\NormalTok{)}
\end{Highlighting}
\end{Shaded}

\end{frame}

\begin{frame}[fragile]{Constructing the market call function (5)}

The data frame \texttt{callstrike.df} collects the call premium-strike
functions for the selected dates:

\begin{Shaded}
\begin{Highlighting}[]
\NormalTok{callstrike.df =}\StringTok{ }\KeywordTok{data.frame}\NormalTok{(K_preBrexit, Call_preBrexit, }
\NormalTok{                           K_Brexit, Call_Brexit, }
\NormalTok{                           K_postBrexit, Call_postBrexit)}
\end{Highlighting}
\end{Shaded}

\end{frame}

\begin{frame}[fragile]{Constructing the market call function (6)}

Plot the market call function

\begin{Shaded}
\begin{Highlighting}[]
\KeywordTok{ggplot}\NormalTok{(callstrike.df) }\OperatorTok{+}
\StringTok{  }\KeywordTok{geom_line}\NormalTok{(}\KeywordTok{aes}\NormalTok{(}\DataTypeTok{x=}\NormalTok{K_preBrexit,}\DataTypeTok{y=}\NormalTok{Call_preBrexit), }\DataTypeTok{col=}\StringTok{"red"}\NormalTok{, }\DataTypeTok{size=}\DecValTok{1}\NormalTok{) }\OperatorTok{+}\StringTok{ }
\StringTok{  }\KeywordTok{geom_line}\NormalTok{(}\KeywordTok{aes}\NormalTok{(}\DataTypeTok{x=}\NormalTok{K_Brexit, }\DataTypeTok{y=}\NormalTok{Call_Brexit), }\DataTypeTok{col=}\StringTok{"darkcyan"}\NormalTok{, }\DataTypeTok{size=}\DecValTok{1}\NormalTok{) }\OperatorTok{+}
\StringTok{  }\KeywordTok{geom_line}\NormalTok{(}\KeywordTok{aes}\NormalTok{(}\DataTypeTok{x=}\NormalTok{K_postBrexit, }\DataTypeTok{y=}\NormalTok{Call_postBrexit), }\DataTypeTok{col=}\StringTok{"blue"}\NormalTok{, }\DataTypeTok{size=}\DecValTok{1}\NormalTok{) }\OperatorTok{+}
\StringTok{  }\KeywordTok{labs}\NormalTok{(}\DataTypeTok{x=}\StringTok{"GBPUSD"}\NormalTok{, }\DataTypeTok{y=}\StringTok{"Strike price"}\NormalTok{) }\OperatorTok{+}\StringTok{ }
\StringTok{  }\KeywordTok{geom_vline}\NormalTok{(}\DataTypeTok{xintercept =}\NormalTok{ spot[}\DecValTok{1}\NormalTok{], }\DataTypeTok{col=}\StringTok{"red"}\NormalTok{, }\DataTypeTok{linetype=}\StringTok{"longdash"}\NormalTok{) }\OperatorTok{+}\StringTok{ }
\StringTok{  }\KeywordTok{geom_vline}\NormalTok{(}\DataTypeTok{xintercept =}\NormalTok{ spot[}\DecValTok{2}\NormalTok{], }\DataTypeTok{col=}\StringTok{"darkcyan"}\NormalTok{, }\DataTypeTok{linetype=}\StringTok{"longdash"}\NormalTok{) }\OperatorTok{+}
\StringTok{  }\KeywordTok{geom_vline}\NormalTok{(}\DataTypeTok{xintercept =}\NormalTok{ spot[}\DecValTok{3}\NormalTok{], }\DataTypeTok{col=}\StringTok{"blue"}\NormalTok{, }\DataTypeTok{linetype=}\StringTok{"longdash"}\NormalTok{) }\OperatorTok{+}
\StringTok{  }\KeywordTok{geom_hline}\NormalTok{(}\DataTypeTok{yintercept =} \DecValTok{0}\NormalTok{, }\DataTypeTok{col=}\StringTok{"black"}\NormalTok{)}
\end{Highlighting}
\end{Shaded}

\end{frame}

\begin{frame}{Constructing the market call function (7)}

\includegraphics[width=0.9\linewidth]{2018_02_07_IMF_FXCourse_files/figure-beamer/unnamed-chunk-44-1}

\end{frame}

\begin{frame}{Constructing the market call function (8)}

Question: the call premium-strike functions are downward sloping,
i.e.~the call premium is higher for lower strike prices. Does this make
sense? Explain why.

\end{frame}

\begin{frame}[fragile]{Constructing the market put function (1)}

\begin{Shaded}
\begin{Highlighting}[]
\NormalTok{Put_preBrexit =}\StringTok{ }\KeywordTok{mapply}\NormalTok{(GKoption.premium, K_preBrexit, }
\NormalTok{                       smile.df}\OperatorTok{$}\NormalTok{preBrexit,}\DataTypeTok{S=}\NormalTok{spot[}\DecValTok{1}\NormalTok{],}\DataTypeTok{Tenor=}\NormalTok{Tenor,}
                       \DataTypeTok{fwd=}\NormalTok{forward[}\DecValTok{1}\NormalTok{],}\DataTypeTok{rf=}\NormalTok{rf[}\DecValTok{1}\NormalTok{],}\DataTypeTok{option_type=}\StringTok{"p"}\NormalTok{)}

\NormalTok{Put_Brexit =}\StringTok{ }\KeywordTok{mapply}\NormalTok{(GKoption.premium, K_Brexit, smile.df}\OperatorTok{$}\NormalTok{Brexit,}
                        \DataTypeTok{S=}\NormalTok{spot[}\DecValTok{2}\NormalTok{],}\DataTypeTok{Tenor=}\NormalTok{Tenor,}\DataTypeTok{fwd=}\NormalTok{forward[}\DecValTok{2}\NormalTok{],}
                        \DataTypeTok{rf=}\NormalTok{rf[}\DecValTok{2}\NormalTok{],}\DataTypeTok{option_type=}\StringTok{"p"}\NormalTok{)}

\NormalTok{Put_postBrexit =}\StringTok{ }\KeywordTok{mapply}\NormalTok{(GKoption.premium, K_postBrexit, }
\NormalTok{                        smile.df}\OperatorTok{$}\NormalTok{postBrexit,}\DataTypeTok{S=}\NormalTok{spot[}\DecValTok{3}\NormalTok{],}\DataTypeTok{Tenor=}\NormalTok{Tenor,}
                        \DataTypeTok{fwd=}\NormalTok{forward[}\DecValTok{3}\NormalTok{],}\DataTypeTok{rf=}\NormalTok{rf[}\DecValTok{3}\NormalTok{],}\DataTypeTok{option_type=}\StringTok{"p"}\NormalTok{)}
\end{Highlighting}
\end{Shaded}

\end{frame}

\begin{frame}[fragile]{Constructing the market put function (2)}

\begin{Shaded}
\begin{Highlighting}[]
\NormalTok{putstrike.df =}\StringTok{ }\KeywordTok{data.frame}\NormalTok{(K_preBrexit, Put_preBrexit, }
\NormalTok{                          K_Brexit, Put_Brexit, }
\NormalTok{                          K_postBrexit, Put_postBrexit)}
\end{Highlighting}
\end{Shaded}

\end{frame}

\begin{frame}[fragile]{Constructing the market put function (3)}

\begin{Shaded}
\begin{Highlighting}[]
\KeywordTok{ggplot}\NormalTok{(putstrike.df) }\OperatorTok{+}\StringTok{ }
\StringTok{  }\KeywordTok{geom_line}\NormalTok{(}\KeywordTok{aes}\NormalTok{(}\DataTypeTok{x=}\NormalTok{K_preBrexit,}\DataTypeTok{y=}\NormalTok{Put_preBrexit), }\DataTypeTok{col=}\StringTok{"red"}\NormalTok{, }\DataTypeTok{size=}\DecValTok{1}\NormalTok{) }\OperatorTok{+}\StringTok{ }
\StringTok{  }\KeywordTok{geom_line}\NormalTok{(}\KeywordTok{aes}\NormalTok{(}\DataTypeTok{x=}\NormalTok{K_Brexit, }\DataTypeTok{y=}\NormalTok{Put_Brexit), }\DataTypeTok{col=}\StringTok{"darkcyan"}\NormalTok{, }\DataTypeTok{size=}\DecValTok{1}\NormalTok{) }\OperatorTok{+}
\StringTok{  }\KeywordTok{geom_line}\NormalTok{(}\KeywordTok{aes}\NormalTok{(}\DataTypeTok{x=}\NormalTok{K_postBrexit, }\DataTypeTok{y=}\NormalTok{Put_postBrexit), }\DataTypeTok{col=}\StringTok{"blue"}\NormalTok{, }\DataTypeTok{size=}\DecValTok{1}\NormalTok{) }\OperatorTok{+}
\StringTok{  }\KeywordTok{labs}\NormalTok{(}\DataTypeTok{x=}\StringTok{"Strike price"}\NormalTok{, }\DataTypeTok{y=}\StringTok{"Put premium"}\NormalTok{) }\OperatorTok{+}\StringTok{ }
\StringTok{  }\KeywordTok{geom_vline}\NormalTok{(}\DataTypeTok{xintercept =}\NormalTok{ spot[}\DecValTok{1}\NormalTok{], }\DataTypeTok{col=}\StringTok{"red"}\NormalTok{, }\DataTypeTok{linetype=}\StringTok{"longdash"}\NormalTok{) }\OperatorTok{+}\StringTok{ }
\StringTok{  }\KeywordTok{geom_vline}\NormalTok{(}\DataTypeTok{xintercept =}\NormalTok{ spot[}\DecValTok{2}\NormalTok{], }\DataTypeTok{col=}\StringTok{"darkcyan"}\NormalTok{, }\DataTypeTok{linetype=}\StringTok{"longdash"}\NormalTok{) }\OperatorTok{+}
\StringTok{  }\KeywordTok{geom_vline}\NormalTok{(}\DataTypeTok{xintercept =}\NormalTok{ spot[}\DecValTok{3}\NormalTok{], }\DataTypeTok{col=}\StringTok{"blue"}\NormalTok{, }\DataTypeTok{linetype=}\StringTok{"longdash"}\NormalTok{) }\OperatorTok{+}
\StringTok{  }\KeywordTok{geom_hline}\NormalTok{(}\DataTypeTok{yintercept =} \DecValTok{0}\NormalTok{, }\DataTypeTok{col=}\StringTok{"black"}\NormalTok{)}
\end{Highlighting}
\end{Shaded}

\end{frame}

\begin{frame}{Constructing the market put function (4)}

\includegraphics[width=0.9\linewidth]{2018_02_07_IMF_FXCourse_files/figure-beamer/unnamed-chunk-48-1}

\end{frame}

\begin{frame}{}

\color{blue} \LARGE{Part F:}\\
\LARGE{Extracting Risk-Neutral Densities}\\
\Large{Generalized Beta Density}

\end{frame}

\begin{frame}[fragile]{Generalized Beta Density (1)}

\begin{itemize}
\tightlist
\item
  Parametric method
\item
  Implemented in package \texttt{RND}
\item
  Requires as inputs

  \begin{itemize}
  \tightlist
  \item
    call premia and strike range
  \item
    put premia and strike range
  \item
    spot rate
  \item
    implied domestic interest rate
  \item
    option tenor
  \end{itemize}
\end{itemize}

\end{frame}

\begin{frame}[fragile]{Generalized Beta Density (2)}

Let's start with the pre-Brexit period

\begin{Shaded}
\begin{Highlighting}[]
\NormalTok{r =}\StringTok{ }\NormalTok{imp_rd[}\DecValTok{1}\NormalTok{]                  }\CommentTok{# domestic interest rate}
\NormalTok{te=}\StringTok{ }\NormalTok{Tenor                      }\CommentTok{# tenor of the option}
\NormalTok{y =}\StringTok{ }\NormalTok{rf[}\DecValTok{1}\NormalTok{]                      }\CommentTok{# foreign interest rate}
\NormalTok{s0=}\StringTok{ }\NormalTok{spot[}\DecValTok{1}\NormalTok{]                    }\CommentTok{# spot exchange rate}
\NormalTok{call.premium =}\StringTok{ }\NormalTok{Call_preBrexit  }\CommentTok{# vector of call premium values}
\NormalTok{call.strikes =}\StringTok{ }\NormalTok{K_preBrexit     }\CommentTok{# vector of corresponding call strikes}
\NormalTok{put.premium =}\StringTok{ }\NormalTok{Put_preBrexit    }\CommentTok{# vector of put premium values}
\NormalTok{put.strikes =}\StringTok{ }\NormalTok{K_preBrexit      }\CommentTok{# vector of corresponding put strikes}
\end{Highlighting}
\end{Shaded}

\end{frame}

\begin{frame}[fragile]{Generalized Beta Density (3)}

Extract the parameter distribution using the function
\texttt{extract.gb.density()}:

\begin{Shaded}
\begin{Highlighting}[]
\KeywordTok{library}\NormalTok{(RND)}
\NormalTok{gb.preBrexit =}\StringTok{ }\KeywordTok{extract.gb.density}\NormalTok{(}\DataTypeTok{initial.values=}\KeywordTok{c}\NormalTok{(}\OtherTok{NA}\NormalTok{,}\OtherTok{NA}\NormalTok{,}\OtherTok{NA}\NormalTok{,}\OtherTok{NA}\NormalTok{), }
                                  \DataTypeTok{r=}\NormalTok{r, }\DataTypeTok{te=}\NormalTok{te, }\DataTypeTok{y=}\NormalTok{y, }\DataTypeTok{s0=}\NormalTok{s0, }
                                  \DataTypeTok{market.calls=}\NormalTok{call.premium, }
                                  \DataTypeTok{call.strikes =}\NormalTok{ call.strikes, }
                                  \DataTypeTok{call.weights =}\DecValTok{1}\NormalTok{,}
                                  \DataTypeTok{market.puts =}\NormalTok{ put.premium, }
                                  \DataTypeTok{put.strikes =}\NormalTok{ put.strikes, }
                                  \DataTypeTok{put.weights =} \DecValTok{1}\NormalTok{,}
                                  \DataTypeTok{lambda=}\DecValTok{1}\NormalTok{, }\DataTypeTok{hessian.flag=}\NormalTok{F)}
\end{Highlighting}
\end{Shaded}

\end{frame}

\begin{frame}[fragile]{Generalized Beta Density (4)}

Obtain the risk neutral distribution by passing a range of strike prices

\begin{Shaded}
\begin{Highlighting}[]
\NormalTok{Krange =}\StringTok{ }\KeywordTok{seq}\NormalTok{(}\FloatTok{0.9}\OperatorTok{*}\KeywordTok{min}\NormalTok{(K_preBrexit, K_Brexit, K_postBrexit), }
             \FloatTok{1.1}\OperatorTok{*}\KeywordTok{max}\NormalTok{(K_preBrexit, K_Brexit, K_postBrexit), }
             \FloatTok{0.01}\NormalTok{)}
\end{Highlighting}
\end{Shaded}

to the function \texttt{dgb()}:

\begin{Shaded}
\begin{Highlighting}[]
\NormalTok{gb =}\StringTok{ }\NormalTok{gb.preBrexit}
\NormalTok{gb.rnd.preBrexit =}\StringTok{ }\KeywordTok{dgb}\NormalTok{(Krange,gb}\OperatorTok{$}\NormalTok{a, gb}\OperatorTok{$}\NormalTok{b, gb}\OperatorTok{$}\NormalTok{v, gb}\OperatorTok{$}\NormalTok{w)}
\end{Highlighting}
\end{Shaded}

\end{frame}

\begin{frame}[fragile]{Generalized Beta Density (5)}

Do the same for the Brexit \ldots{}

\begin{Shaded}
\begin{Highlighting}[]
\NormalTok{r =}\StringTok{ }\NormalTok{imp_rd[}\DecValTok{2}\NormalTok{]}
\NormalTok{y =}\StringTok{ }\NormalTok{rf[}\DecValTok{2}\NormalTok{]}
\NormalTok{s0=}\StringTok{ }\NormalTok{spot[}\DecValTok{2}\NormalTok{]}
\NormalTok{call.premium =}\StringTok{ }\NormalTok{Call_Brexit}
\NormalTok{call.strikes =}\StringTok{ }\NormalTok{K_Brexit}
\NormalTok{put.premium =}\StringTok{ }\NormalTok{Put_Brexit}
\NormalTok{put.strikes =}\StringTok{ }\NormalTok{K_Brexit}
\end{Highlighting}
\end{Shaded}

\end{frame}

\begin{frame}[fragile]{Generalized Beta Density (6)}

\begin{Shaded}
\begin{Highlighting}[]
\NormalTok{gb.Brexit =}\StringTok{ }\KeywordTok{extract.gb.density}\NormalTok{(}\DataTypeTok{initial.values=}\KeywordTok{c}\NormalTok{(}\OtherTok{NA}\NormalTok{,}\OtherTok{NA}\NormalTok{,}\OtherTok{NA}\NormalTok{,}\OtherTok{NA}\NormalTok{), }
                       \DataTypeTok{r=}\NormalTok{r, }\DataTypeTok{te=}\NormalTok{te, }\DataTypeTok{y=}\NormalTok{y, }\DataTypeTok{s0=}\NormalTok{s0, }
                       \DataTypeTok{market.calls=}\NormalTok{call.premium, }
                       \DataTypeTok{call.strikes =}\NormalTok{ call.strikes, }
                       \DataTypeTok{call.weights =}\DecValTok{1}\NormalTok{,}
                       \DataTypeTok{market.puts =}\NormalTok{ put.premium, }
                       \DataTypeTok{put.strikes =}\NormalTok{ put.strikes, }
                       \DataTypeTok{put.weights =} \DecValTok{1}\NormalTok{,  }
                       \DataTypeTok{lambda=}\DecValTok{1}\NormalTok{, }\DataTypeTok{hessian.flag=}\NormalTok{F)}
\NormalTok{gb =}\StringTok{ }\NormalTok{gb.Brexit}
\NormalTok{gb.rnd.Brexit =}\StringTok{ }\KeywordTok{dgb}\NormalTok{(Krange,gb}\OperatorTok{$}\NormalTok{a, gb}\OperatorTok{$}\NormalTok{b, gb}\OperatorTok{$}\NormalTok{v, gb}\OperatorTok{$}\NormalTok{w)}
\end{Highlighting}
\end{Shaded}

\end{frame}

\begin{frame}[fragile]{Generalized Beta Density (7)}

And more of the same for post-Brexit:

\begin{Shaded}
\begin{Highlighting}[]
\NormalTok{r =}\StringTok{ }\NormalTok{imp_rd[}\DecValTok{3}\NormalTok{]}
\NormalTok{y =}\StringTok{ }\NormalTok{rf[}\DecValTok{3}\NormalTok{]}
\NormalTok{s0=}\StringTok{ }\NormalTok{spot[}\DecValTok{3}\NormalTok{]}
\NormalTok{call.premium =}\StringTok{ }\NormalTok{Call_postBrexit}
\NormalTok{call.strikes =}\StringTok{ }\NormalTok{K_postBrexit}
\NormalTok{put.premium =}\StringTok{ }\NormalTok{Put_postBrexit}
\NormalTok{put.strikes =}\StringTok{ }\NormalTok{K_postBrexit}
\end{Highlighting}
\end{Shaded}

\end{frame}

\begin{frame}[fragile]{Generalized Beta Density (8)}

\begin{Shaded}
\begin{Highlighting}[]
\NormalTok{gb.postBrexit =}\StringTok{ }\KeywordTok{extract.gb.density}\NormalTok{(}\DataTypeTok{initial.values=}\KeywordTok{c}\NormalTok{(}\OtherTok{NA}\NormalTok{,}\OtherTok{NA}\NormalTok{,}\OtherTok{NA}\NormalTok{,}\OtherTok{NA}\NormalTok{), }
                       \DataTypeTok{r=}\NormalTok{r, }\DataTypeTok{te=}\NormalTok{te, }\DataTypeTok{y=}\NormalTok{y, }\DataTypeTok{s0=}\NormalTok{s0, }
                       \DataTypeTok{market.calls=}\NormalTok{call.premium, }
                       \DataTypeTok{call.strikes =}\NormalTok{ call.strikes, }
                       \DataTypeTok{call.weights =}\DecValTok{1}\NormalTok{,}
                       \DataTypeTok{market.puts =}\NormalTok{ put.premium, }
                       \DataTypeTok{put.strikes =}\NormalTok{ put.strikes, }
                       \DataTypeTok{put.weights =} \DecValTok{1}\NormalTok{,  }
                       \DataTypeTok{lambda=}\DecValTok{1}\NormalTok{, }\DataTypeTok{hessian.flag=}\NormalTok{F)}
\NormalTok{gb =}\StringTok{ }\NormalTok{gb.postBrexit}
\NormalTok{gb.rnd.postBrexit =}\StringTok{ }\KeywordTok{dgb}\NormalTok{(Krange,gb}\OperatorTok{$}\NormalTok{a, gb}\OperatorTok{$}\NormalTok{b, gb}\OperatorTok{$}\NormalTok{v, gb}\OperatorTok{$}\NormalTok{w)}
\end{Highlighting}
\end{Shaded}

\end{frame}

\begin{frame}[fragile]{Generalized Beta Density (9)}

\begin{Shaded}
\begin{Highlighting}[]
\NormalTok{gb.rnd.df =}\StringTok{ }\KeywordTok{data.frame}\NormalTok{(Krange,gb.rnd.preBrexit, }
\NormalTok{                       gb.rnd.Brexit, gb.rnd.postBrexit)}

\KeywordTok{ggplot}\NormalTok{(}\DataTypeTok{data=}\NormalTok{gb.rnd.df, }\KeywordTok{aes}\NormalTok{(}\DataTypeTok{x=}\NormalTok{Krange)) }\OperatorTok{+}\StringTok{ }
\StringTok{  }\KeywordTok{geom_line}\NormalTok{(}\KeywordTok{aes}\NormalTok{(}\DataTypeTok{y=}\NormalTok{gb.rnd.preBrexit), }\DataTypeTok{col=}\StringTok{"red"}\NormalTok{, }\DataTypeTok{size=}\FloatTok{1.25}\NormalTok{) }\OperatorTok{+}
\StringTok{  }\KeywordTok{geom_line}\NormalTok{(}\KeywordTok{aes}\NormalTok{(}\DataTypeTok{y=}\NormalTok{gb.rnd.Brexit), }\DataTypeTok{col=}\StringTok{"darkcyan"}\NormalTok{, }\DataTypeTok{size=}\FloatTok{1.25}\NormalTok{) }\OperatorTok{+}
\StringTok{  }\KeywordTok{geom_line}\NormalTok{(}\KeywordTok{aes}\NormalTok{(}\DataTypeTok{y=}\NormalTok{gb.rnd.postBrexit), }\DataTypeTok{col=}\StringTok{"blue"}\NormalTok{, }\DataTypeTok{size=}\FloatTok{1.25}\NormalTok{) }\OperatorTok{+}
\StringTok{  }\KeywordTok{geom_vline}\NormalTok{(}\DataTypeTok{xintercept =}\NormalTok{ spot[}\DecValTok{1}\NormalTok{], }\DataTypeTok{col=}\StringTok{"red"}\NormalTok{, }\DataTypeTok{linetype=}\StringTok{"longdash"}\NormalTok{) }\OperatorTok{+}\StringTok{ }
\StringTok{  }\KeywordTok{geom_vline}\NormalTok{(}\DataTypeTok{xintercept =}\NormalTok{ spot[}\DecValTok{2}\NormalTok{], }\DataTypeTok{col=}\StringTok{"darkcyan"}\NormalTok{, }\DataTypeTok{linetype=}\StringTok{"longdash"}\NormalTok{) }\OperatorTok{+}
\StringTok{  }\KeywordTok{geom_vline}\NormalTok{(}\DataTypeTok{xintercept =}\NormalTok{ spot[}\DecValTok{3}\NormalTok{], }\DataTypeTok{col=}\StringTok{"blue"}\NormalTok{, }\DataTypeTok{linetype=}\StringTok{"longdash"}\NormalTok{) }\OperatorTok{+}
\StringTok{  }\KeywordTok{geom_hline}\NormalTok{(}\DataTypeTok{yintercept=}\DecValTok{0}\NormalTok{, }\DataTypeTok{col=}\StringTok{"black"}\NormalTok{, }\DataTypeTok{size=}\FloatTok{0.5}\NormalTok{) }\OperatorTok{+}
\StringTok{  }\KeywordTok{labs}\NormalTok{(}\DataTypeTok{x=}\StringTok{"GBPUSD"}\NormalTok{, }\DataTypeTok{y=}\StringTok{"3-month risk-neutral density"}\NormalTok{)   }
\end{Highlighting}
\end{Shaded}

\end{frame}

\begin{frame}{Generalized Beta Density (10)}

\begin{figure}
\includegraphics[width=1\linewidth]{2018_02_07_IMF_FXCourse_files/figure-beamer/unnamed-chunk-58-1} \caption{Generalized beta risk neutral distributions. Pre-Brexit: red; Brexit: cyan; post-Brexit: blue}\label{fig:unnamed-chunk-58}
\end{figure}

\end{frame}

\begin{frame}{}

\color{blue} \LARGE{Part F:}\\
\LARGE{Extracting Risk-Neutral Densities}\\
\Large{Edgeworth expansion}

\end{frame}

\begin{frame}[fragile]{Edgeworth expansion (1)}

\begin{itemize}
\tightlist
\item
  Semi-parametric method
\item
  Implemented in package \texttt{RND}
\item
  Requires as inputs

  \begin{itemize}
  \tightlist
  \item
    call premia and strike range
  \item
    spot rate
  \item
    implied domestic interest rate
  \item
    option tenor
  \end{itemize}
\end{itemize}

\end{frame}

\begin{frame}[fragile]{Edgeworth expansion (2)}

Calculations in Brexit period

\begin{Shaded}
\begin{Highlighting}[]
\NormalTok{r =}\StringTok{ }\NormalTok{imp_rd[}\DecValTok{1}\NormalTok{]                      }\CommentTok{# domestic interest rate}
\NormalTok{te=}\StringTok{ }\NormalTok{Tenor                          }\CommentTok{# tenor of the option}
\NormalTok{y =}\StringTok{ }\NormalTok{rf[}\DecValTok{1}\NormalTok{]                          }\CommentTok{# foreign interest rate}
\NormalTok{s0=}\StringTok{ }\NormalTok{spot[}\DecValTok{1}\NormalTok{]                        }\CommentTok{# spot exchange rate}
\NormalTok{call.premium =}\StringTok{ }\NormalTok{Call_preBrexit      }\CommentTok{# vector of call premium values}
\NormalTok{call.strikes =}\StringTok{ }\NormalTok{K_preBrexit         }\CommentTok{# vector of corresponding call strikes}
\end{Highlighting}
\end{Shaded}

\end{frame}

\begin{frame}[fragile]{Edgeworth expansion (3)}

The function \texttt{extract.ew.density()} calculates the parameters of
the Edgeworth expansion:

\begin{Shaded}
\begin{Highlighting}[]
\NormalTok{ew.preBrexit =}\StringTok{ }\KeywordTok{extract.ew.density}\NormalTok{(}\DataTypeTok{initial.values =} \KeywordTok{rep}\NormalTok{(}\OtherTok{NA}\NormalTok{,}\DecValTok{2}\NormalTok{), }
                       \DataTypeTok{r=}\NormalTok{r, }\DataTypeTok{y=}\NormalTok{y, }\DataTypeTok{te=}\NormalTok{te, }\DataTypeTok{s0=}\NormalTok{s0, }
                       \DataTypeTok{market.calls=}\NormalTok{call.premium, }
                       \DataTypeTok{call.strikes =}\NormalTok{ call.strikes, }
                       \DataTypeTok{call.weights =}\DecValTok{1}\NormalTok{, }\DataTypeTok{lambda=}\DecValTok{1}\NormalTok{, }\DataTypeTok{hessian.flag=}\NormalTok{F, }
                       \DataTypeTok{cl =} \KeywordTok{list}\NormalTok{(}\DataTypeTok{maxit=}\DecValTok{10000}\NormalTok{))}
\end{Highlighting}
\end{Shaded}

which we then input into the function \texttt{dew()} to obtain the
distribution.

\begin{Shaded}
\begin{Highlighting}[]
\NormalTok{ew =}\StringTok{ }\NormalTok{ew.preBrexit}
\NormalTok{ew.rnd.preBrexit  =}\StringTok{ }\KeywordTok{dew}\NormalTok{(Krange,r,y,te,s0,}
\NormalTok{                        ew}\OperatorTok{$}\NormalTok{sigma, ew}\OperatorTok{$}\NormalTok{skew, ew}\OperatorTok{$}\NormalTok{kurt)}
\end{Highlighting}
\end{Shaded}

\end{frame}

\begin{frame}[fragile]{Edgeworth expansion (4)}

Repeat the recipe for Brexit \ldots{}

\begin{Shaded}
\begin{Highlighting}[]
\NormalTok{r =}\StringTok{ }\NormalTok{imp_rd[}\DecValTok{2}\NormalTok{]                      }\CommentTok{# domestic interest rate}
\NormalTok{te=}\StringTok{ }\NormalTok{Tenor                          }\CommentTok{# tenor of the option}
\NormalTok{y =}\StringTok{ }\NormalTok{rf[}\DecValTok{2}\NormalTok{]                          }\CommentTok{# foreign interest rate}
\NormalTok{s0=}\StringTok{ }\NormalTok{spot[}\DecValTok{2}\NormalTok{]                        }\CommentTok{# spot exchange rate}
\NormalTok{call.premium =}\StringTok{ }\NormalTok{Call_Brexit         }\CommentTok{# vector of call premium values}
\NormalTok{call.strikes =}\StringTok{ }\NormalTok{K_Brexit            }\CommentTok{# vector of corresponding call strikes}
\end{Highlighting}
\end{Shaded}

\end{frame}

\begin{frame}[fragile]{Edgeworth expansion (5)}

\begin{Shaded}
\begin{Highlighting}[]
\NormalTok{ew.Brexit =}\StringTok{ }\KeywordTok{extract.ew.density}\NormalTok{(}\DataTypeTok{initial.values =} \KeywordTok{rep}\NormalTok{(}\OtherTok{NA}\NormalTok{,}\DecValTok{2}\NormalTok{), }
                    \DataTypeTok{r=}\NormalTok{r, }\DataTypeTok{y=}\NormalTok{y, }\DataTypeTok{te=}\NormalTok{te, }\DataTypeTok{s0=}\NormalTok{s0, }
                    \DataTypeTok{market.calls=}\NormalTok{call.premium, }
                    \DataTypeTok{call.strikes =}\NormalTok{ call.strikes, }
                    \DataTypeTok{call.weights =}\DecValTok{1}\NormalTok{, }\DataTypeTok{lambda=}\DecValTok{1}\NormalTok{, }\DataTypeTok{hessian.flag=}\NormalTok{F,}
                    \DataTypeTok{cl =} \KeywordTok{list}\NormalTok{(}\DataTypeTok{maxit=}\DecValTok{10000}\NormalTok{))}
\NormalTok{ew =}\StringTok{ }\NormalTok{ew.Brexit}
\NormalTok{ew.rnd.Brexit  =}\StringTok{ }\KeywordTok{dew}\NormalTok{(Krange,r,y,te,s0,ew}\OperatorTok{$}\NormalTok{sigma, ew}\OperatorTok{$}\NormalTok{skew, ew}\OperatorTok{$}\NormalTok{kurt)}
\end{Highlighting}
\end{Shaded}

\end{frame}

\begin{frame}[fragile]{Edgeworth expansion (6)}

And for post-Brexit

\begin{Shaded}
\begin{Highlighting}[]
\NormalTok{r =}\StringTok{ }\NormalTok{imp_rd[}\DecValTok{3}\NormalTok{]                    }\CommentTok{# domestic interest rate}
\NormalTok{te=}\StringTok{ }\NormalTok{Tenor                        }\CommentTok{# tenor of the option}
\NormalTok{y =}\StringTok{ }\NormalTok{rf[}\DecValTok{3}\NormalTok{]                        }\CommentTok{# foreign interest rate}
\NormalTok{s0=}\StringTok{ }\NormalTok{spot[}\DecValTok{3}\NormalTok{]                      }\CommentTok{# spot exchange rate}
\NormalTok{call.premium =}\StringTok{ }\NormalTok{Call_postBrexit   }\CommentTok{# vector of call premium values}
\NormalTok{call.strikes =}\StringTok{ }\NormalTok{K_postBrexit      }\CommentTok{# vector of corresponding call strikes}
\end{Highlighting}
\end{Shaded}

\end{frame}

\begin{frame}[fragile]{Edgeworth expansion (7)}

\begin{Shaded}
\begin{Highlighting}[]
\NormalTok{ew.postBrexit =}\StringTok{ }\KeywordTok{extract.ew.density}\NormalTok{(}\DataTypeTok{initial.values =} \KeywordTok{rep}\NormalTok{(}\OtherTok{NA}\NormalTok{,}\DecValTok{2}\NormalTok{), }
                    \DataTypeTok{r=}\NormalTok{r, }\DataTypeTok{y=}\NormalTok{y, }\DataTypeTok{te=}\NormalTok{te, }\DataTypeTok{s0=}\NormalTok{s0, }
                    \DataTypeTok{market.calls=}\NormalTok{call.premium, }
                    \DataTypeTok{call.strikes =}\NormalTok{ call.strikes, }
                    \DataTypeTok{call.weights =}\DecValTok{1}\NormalTok{, }\DataTypeTok{lambda=}\DecValTok{1}\NormalTok{, }\DataTypeTok{hessian.flag=}\NormalTok{F,}
                    \DataTypeTok{cl =} \KeywordTok{list}\NormalTok{(}\DataTypeTok{maxit=}\DecValTok{10000}\NormalTok{))}
\NormalTok{ew =}\StringTok{ }\NormalTok{ew.postBrexit}
\NormalTok{ew.rnd.postBrexit  =}\StringTok{ }\KeywordTok{dew}\NormalTok{(Krange,r,y,te,s0,ew}\OperatorTok{$}\NormalTok{sigma, ew}\OperatorTok{$}\NormalTok{skew, ew}\OperatorTok{$}\NormalTok{kurt)}
\end{Highlighting}
\end{Shaded}

\end{frame}

\begin{frame}[fragile]{Edgeworth expansion (8)}

Produce a pretty chart

\begin{Shaded}
\begin{Highlighting}[]
\NormalTok{ew.rnd.df =}\StringTok{ }\KeywordTok{data.frame}\NormalTok{(Krange, ew.rnd.preBrexit, }
\NormalTok{                       ew.rnd.Brexit, ew.rnd.postBrexit)}

\KeywordTok{ggplot}\NormalTok{(}\DataTypeTok{data=}\NormalTok{ew.rnd.df, }\KeywordTok{aes}\NormalTok{(}\DataTypeTok{x=}\NormalTok{Krange)) }\OperatorTok{+}\StringTok{ }
\StringTok{  }\KeywordTok{geom_line}\NormalTok{(}\KeywordTok{aes}\NormalTok{(}\DataTypeTok{y=}\NormalTok{ew.rnd.preBrexit), }\DataTypeTok{col=}\StringTok{"red"}\NormalTok{, }\DataTypeTok{size=}\FloatTok{1.25}\NormalTok{) }\OperatorTok{+}
\StringTok{  }\KeywordTok{geom_line}\NormalTok{(}\KeywordTok{aes}\NormalTok{(}\DataTypeTok{y=}\NormalTok{ew.rnd.Brexit), }\DataTypeTok{col=}\StringTok{"darkcyan"}\NormalTok{, }\DataTypeTok{size=}\FloatTok{1.25}\NormalTok{) }\OperatorTok{+}
\StringTok{  }\KeywordTok{geom_line}\NormalTok{(}\KeywordTok{aes}\NormalTok{(}\DataTypeTok{y=}\NormalTok{ew.rnd.postBrexit), }\DataTypeTok{col=}\StringTok{"blue"}\NormalTok{, }\DataTypeTok{size=}\FloatTok{1.25}\NormalTok{) }\OperatorTok{+}
\StringTok{  }\KeywordTok{geom_vline}\NormalTok{(}\DataTypeTok{xintercept =}\NormalTok{ spot[}\DecValTok{1}\NormalTok{], }\DataTypeTok{col=}\StringTok{"red"}\NormalTok{, }\DataTypeTok{linetype=}\StringTok{"longdash"}\NormalTok{) }\OperatorTok{+}\StringTok{ }
\StringTok{  }\KeywordTok{geom_vline}\NormalTok{(}\DataTypeTok{xintercept =}\NormalTok{ spot[}\DecValTok{2}\NormalTok{], }\DataTypeTok{col=}\StringTok{"darkcyan"}\NormalTok{, }\DataTypeTok{linetype=}\StringTok{"longdash"}\NormalTok{) }\OperatorTok{+}
\StringTok{  }\KeywordTok{geom_vline}\NormalTok{(}\DataTypeTok{xintercept =}\NormalTok{ spot[}\DecValTok{3}\NormalTok{], }\DataTypeTok{col=}\StringTok{"blue"}\NormalTok{, }\DataTypeTok{linetype=}\StringTok{"longdash"}\NormalTok{) }\OperatorTok{+}
\StringTok{  }\KeywordTok{geom_hline}\NormalTok{(}\DataTypeTok{yintercept=}\DecValTok{0}\NormalTok{, }\DataTypeTok{col=}\StringTok{"black"}\NormalTok{, }\DataTypeTok{size=}\FloatTok{0.5}\NormalTok{) }\OperatorTok{+}
\StringTok{  }\KeywordTok{labs}\NormalTok{(}\DataTypeTok{x=}\StringTok{"GBPUSD"}\NormalTok{, }\DataTypeTok{y=}\StringTok{"3-month risk-neutral density"}\NormalTok{) }
\end{Highlighting}
\end{Shaded}

\end{frame}

\begin{frame}{Edgeworth expansion (9)}

\begin{figure}
\includegraphics[width=1\linewidth]{2018_02_07_IMF_FXCourse_files/figure-beamer/unnamed-chunk-67-1} \caption{Edgeworth expansion risk neutral distributions. Pre-Brexit: red; Brexit: cyan; post-Brexit: blue}\label{fig:unnamed-chunk-67}
\end{figure}

\end{frame}

\begin{frame}{}

\color{blue} \LARGE{Part F:}\\
\LARGE{Extracting Risk-Neutral Densities}\\
\Large{Shimko method}

\end{frame}

\begin{frame}[fragile]{Shimko method (1)}

\begin{itemize}
\tightlist
\item
  Nonparametric method
\item
  Implemented in package \texttt{RND}
\item
  Method requires a volatility smile in the \(\sigma\) - \(K\) space
\end{itemize}

\end{frame}

\begin{frame}[fragile]{Shimko method (2)}

Construct the volatility smile in the \(\sigma\) - \(K\) in the Brexit
period

\begin{Shaded}
\begin{Highlighting}[]
\NormalTok{vol.preBrexit =}\StringTok{ }\NormalTok{vol.data[}\DecValTok{1}\OperatorTok{:}\DecValTok{5}\NormalTok{,}\DecValTok{3}\NormalTok{]         }\CommentTok{# obtain implied volatility}
\NormalTok{delta.shimko =}\StringTok{ }\NormalTok{vol.data[}\DecValTok{1}\OperatorTok{:}\DecValTok{5}\NormalTok{,}\DecValTok{1}\NormalTok{]          }\CommentTok{# obtain the deltas}

\CommentTok{# Calculate the strike prices corresponding to the observed deltas}

\NormalTok{Kshimko_preBrexit =}\StringTok{ }\KeywordTok{mapply}\NormalTok{(get.strike,vol.preBrexit, }
\NormalTok{                           delta.shimko, }\DataTypeTok{S0=}\NormalTok{spot[}\DecValTok{1}\NormalTok{], }
                           \DataTypeTok{fwd=}\NormalTok{forward[}\DecValTok{1}\NormalTok{], }\DataTypeTok{rf=}\NormalTok{rf[}\DecValTok{1}\NormalTok{], }\DataTypeTok{Tenor=}\FloatTok{0.25}\NormalTok{)}

\CommentTok{# Calculate the call premium}

\NormalTok{shimko_preBrexit =}\StringTok{ }\KeywordTok{mapply}\NormalTok{(GKoption.premium, Kshimko_preBrexit, }
\NormalTok{                          vol.preBrexit,}\DataTypeTok{S=}\NormalTok{spot[}\DecValTok{1}\NormalTok{],}\DataTypeTok{Tenor=}\NormalTok{Tenor,}
                          \DataTypeTok{fwd=}\NormalTok{forward[}\DecValTok{1}\NormalTok{],}\DataTypeTok{rf=}\NormalTok{rf[}\DecValTok{1}\NormalTok{],}\DataTypeTok{option_type=}\StringTok{"c"}\NormalTok{)}
\end{Highlighting}
\end{Shaded}

\end{frame}

\begin{frame}[fragile]{Shimko method (3)}

The function in the \texttt{RND} package that calculates the parameters
of the Shimko's quadratic approximation to the volatility smile is
\texttt{extract.shimko.density()}:

\begin{Shaded}
\begin{Highlighting}[]
\CommentTok{# Inputs for extract.shimko.density()}

\NormalTok{r =}\StringTok{ }\NormalTok{imp_rd[}\DecValTok{1}\NormalTok{]                      }\CommentTok{# implied doomestic rate}
\NormalTok{te=}\StringTok{ }\NormalTok{Tenor                          }\CommentTok{# time to  maturity}
\NormalTok{y =}\StringTok{ }\NormalTok{rf[}\DecValTok{1}\NormalTok{]                          }\CommentTok{# foreign interest rate}
\NormalTok{s0=}\StringTok{ }\NormalTok{spot[}\DecValTok{1}\NormalTok{]                        }\CommentTok{# spot exchange rate}
\NormalTok{call.premium =}\StringTok{ }\NormalTok{shimko_preBrexit    }\CommentTok{# call premia values}
\NormalTok{call.strikes =}\StringTok{ }\NormalTok{Kshimko_preBrexit   }\CommentTok{# option strikes}
\NormalTok{b=r}\OperatorTok{-}\NormalTok{y}

\NormalTok{shimko.preBrexit =}\StringTok{ }\KeywordTok{extract.shimko.density}\NormalTok{(}\DataTypeTok{market.calls=}\NormalTok{call.premium, }
                                          \DataTypeTok{call.strikes =}\NormalTok{ call.strikes,}
                                          \DataTypeTok{r=}\NormalTok{r, }\DataTypeTok{y=}\NormalTok{b, }\DataTypeTok{t=}\NormalTok{te, }\DataTypeTok{s0=}\NormalTok{s0, }
                                          \DataTypeTok{lower=}\DecValTok{0}\NormalTok{, }\DataTypeTok{upper=} \DecValTok{30}\NormalTok{)}
\end{Highlighting}
\end{Shaded}

\end{frame}

\begin{frame}[fragile]{Shimko method (4)}

The parameters are then fed to the function \texttt{dshimko()} to obtain
the risk neutral distribution for a wider range of strikes,
\texttt{Krange}:

\begin{Shaded}
\begin{Highlighting}[]
\NormalTok{shimko =}\StringTok{ }\NormalTok{shimko.preBrexit  }

\NormalTok{shimko.rnd.preBrexit =}\StringTok{ }\KeywordTok{dshimko}\NormalTok{(}\DataTypeTok{r=}\NormalTok{r, }\DataTypeTok{te=}\NormalTok{Tenor, }
                               \DataTypeTok{s0=}\NormalTok{s0, }\DataTypeTok{k=}\NormalTok{Krange, }\DataTypeTok{y=}\NormalTok{y,}
                               \DataTypeTok{a0=}\NormalTok{shimko[[}\DecValTok{1}\NormalTok{]]}\OperatorTok{$}\NormalTok{a0, }\DataTypeTok{a1=}\NormalTok{shimko[[}\DecValTok{1}\NormalTok{]]}\OperatorTok{$}\NormalTok{a1, }
                               \DataTypeTok{a2=}\NormalTok{shimko[[}\DecValTok{1}\NormalTok{]]}\OperatorTok{$}\NormalTok{a2)}
\end{Highlighting}
\end{Shaded}

\end{frame}

\begin{frame}[fragile]{Shimko method (5)}

Repeat for Brexit

\begin{Shaded}
\begin{Highlighting}[]
\NormalTok{vol.Brexit =}\StringTok{ }\NormalTok{vol.data[}\DecValTok{6}\OperatorTok{:}\DecValTok{10}\NormalTok{,}\DecValTok{3}\NormalTok{]         }\CommentTok{# obtain implied volatility}

\NormalTok{Kshimko_Brexit =}\StringTok{ }\KeywordTok{mapply}\NormalTok{(get.strike,vol.Brexit, }
\NormalTok{                        delta.shimko,}\DataTypeTok{S0=}\NormalTok{spot[}\DecValTok{2}\NormalTok{], }
                        \DataTypeTok{fwd=}\NormalTok{forward[}\DecValTok{2}\NormalTok{], }\DataTypeTok{rf=}\NormalTok{rf[}\DecValTok{2}\NormalTok{], }\DataTypeTok{Tenor=}\FloatTok{0.25}\NormalTok{)}

\NormalTok{shimko_Brexit =}\StringTok{ }\KeywordTok{mapply}\NormalTok{(GKoption.premium, Kshimko_Brexit, }
\NormalTok{                       vol.Brexit,}\DataTypeTok{S=}\NormalTok{spot[}\DecValTok{2}\NormalTok{],}\DataTypeTok{Tenor=}\NormalTok{Tenor,}
                       \DataTypeTok{fwd=}\NormalTok{forward[}\DecValTok{2}\NormalTok{],}\DataTypeTok{rf=}\NormalTok{rf[}\DecValTok{2}\NormalTok{],}\DataTypeTok{option_type=}\StringTok{"c"}\NormalTok{)}
\end{Highlighting}
\end{Shaded}

\end{frame}

\begin{frame}[fragile]{Shimko method (6)}

\begin{Shaded}
\begin{Highlighting}[]
\NormalTok{r =}\StringTok{ }\NormalTok{imp_rd[}\DecValTok{2}\NormalTok{]}
\NormalTok{y =}\StringTok{ }\NormalTok{rf[}\DecValTok{2}\NormalTok{]}
\NormalTok{s0=}\StringTok{ }\NormalTok{spot[}\DecValTok{2}\NormalTok{]}
\NormalTok{call.premium =}\StringTok{ }\NormalTok{shimko_Brexit}
\NormalTok{call.strikes =}\StringTok{ }\NormalTok{Kshimko_Brexit}
\NormalTok{b=r}\OperatorTok{-}\NormalTok{y}

\NormalTok{shimko.Brexit =}\StringTok{ }\KeywordTok{extract.shimko.density}\NormalTok{(}\DataTypeTok{market.calls=}\NormalTok{call.premium, }
                                       \DataTypeTok{call.strikes =}\NormalTok{ call.strikes,}
                                       \DataTypeTok{r=}\NormalTok{r, }\DataTypeTok{y=}\NormalTok{b, }\DataTypeTok{t=}\NormalTok{te, }\DataTypeTok{s0=}\NormalTok{s0, }
                                       \DataTypeTok{lower=}\DecValTok{0}\NormalTok{, }\DataTypeTok{upper=} \DecValTok{30}\NormalTok{)}

\NormalTok{shimko =}\StringTok{ }\NormalTok{shimko.Brexit}
\NormalTok{shimko.rnd.Brexit =}\StringTok{ }\KeywordTok{dshimko}\NormalTok{(}\DataTypeTok{r=}\NormalTok{r, }\DataTypeTok{te=}\NormalTok{Tenor, }\DataTypeTok{s0=}\NormalTok{s0, }\DataTypeTok{k=}\NormalTok{Krange, }\DataTypeTok{y=}\NormalTok{y,}
                            \DataTypeTok{a0=}\NormalTok{shimko[[}\DecValTok{1}\NormalTok{]]}\OperatorTok{$}\NormalTok{a0, }\DataTypeTok{a1=}\NormalTok{shimko[[}\DecValTok{1}\NormalTok{]]}\OperatorTok{$}\NormalTok{a1, }
                            \DataTypeTok{a2=}\NormalTok{shimko[[}\DecValTok{1}\NormalTok{]]}\OperatorTok{$}\NormalTok{a2)}
\end{Highlighting}
\end{Shaded}

\end{frame}

\begin{frame}[fragile]{Shimko method (7)}

Once more for post-Brexit

\begin{Shaded}
\begin{Highlighting}[]
\NormalTok{vol.postBrexit =}\StringTok{ }\NormalTok{vol.data[}\DecValTok{11}\OperatorTok{:}\DecValTok{15}\NormalTok{,}\DecValTok{3}\NormalTok{]         }\CommentTok{# obtain implied volatility}

\NormalTok{Kshimko_postBrexit =}\StringTok{ }\KeywordTok{mapply}\NormalTok{(get.strike,vol.postBrexit, }
\NormalTok{                        delta.shimko,}\DataTypeTok{S0=}\NormalTok{spot[}\DecValTok{3}\NormalTok{], }
                        \DataTypeTok{fwd=}\NormalTok{forward[}\DecValTok{3}\NormalTok{], }\DataTypeTok{rf=}\NormalTok{rf[}\DecValTok{3}\NormalTok{], }\DataTypeTok{Tenor=}\FloatTok{0.25}\NormalTok{)}

\NormalTok{shimko_postBrexit =}\StringTok{ }\KeywordTok{mapply}\NormalTok{(GKoption.premium, Kshimko_postBrexit, }
\NormalTok{                       vol.postBrexit,}\DataTypeTok{S=}\NormalTok{spot[}\DecValTok{3}\NormalTok{],}\DataTypeTok{Tenor=}\NormalTok{Tenor,}
                       \DataTypeTok{fwd=}\NormalTok{forward[}\DecValTok{3}\NormalTok{],}\DataTypeTok{rf=}\NormalTok{rf[}\DecValTok{3}\NormalTok{],}\DataTypeTok{option_type=}\StringTok{"c"}\NormalTok{)}
\end{Highlighting}
\end{Shaded}

\end{frame}

\begin{frame}[fragile]{Shimko method (8)}

\begin{Shaded}
\begin{Highlighting}[]
\NormalTok{r =}\StringTok{ }\NormalTok{imp_rd[}\DecValTok{3}\NormalTok{]}
\NormalTok{y =}\StringTok{ }\NormalTok{rf[}\DecValTok{3}\NormalTok{]}
\NormalTok{s0=}\StringTok{ }\NormalTok{spot[}\DecValTok{3}\NormalTok{]}
\NormalTok{call.premium =}\StringTok{ }\NormalTok{shimko_postBrexit}
\NormalTok{call.strikes =}\StringTok{ }\NormalTok{Kshimko_postBrexit}
\NormalTok{b=r}\OperatorTok{-}\NormalTok{y}

\NormalTok{shimko.postBrexit =}\StringTok{ }\KeywordTok{extract.shimko.density}\NormalTok{(}\DataTypeTok{market.calls=}\NormalTok{call.premium, }
                                       \DataTypeTok{call.strikes =}\NormalTok{ call.strikes,}
                                       \DataTypeTok{r=}\NormalTok{r, }\DataTypeTok{y=}\NormalTok{b, }\DataTypeTok{t=}\NormalTok{te, }\DataTypeTok{s0=}\NormalTok{s0, }
                                       \DataTypeTok{lower=}\DecValTok{0}\NormalTok{, }\DataTypeTok{upper=} \DecValTok{30}\NormalTok{)}

\NormalTok{shimko =}\StringTok{ }\NormalTok{shimko.postBrexit}
\NormalTok{shimko.rnd.postBrexit =}\StringTok{ }\KeywordTok{dshimko}\NormalTok{(}\DataTypeTok{r=}\NormalTok{r, }\DataTypeTok{te=}\NormalTok{Tenor, }\DataTypeTok{s0=}\NormalTok{s0, }\DataTypeTok{k=}\NormalTok{Krange, }\DataTypeTok{y=}\NormalTok{y,}
                            \DataTypeTok{a0=}\NormalTok{shimko[[}\DecValTok{1}\NormalTok{]]}\OperatorTok{$}\NormalTok{a0, }\DataTypeTok{a1=}\NormalTok{shimko[[}\DecValTok{1}\NormalTok{]]}\OperatorTok{$}\NormalTok{a1, }
                            \DataTypeTok{a2=}\NormalTok{shimko[[}\DecValTok{1}\NormalTok{]]}\OperatorTok{$}\NormalTok{a2)}
\end{Highlighting}
\end{Shaded}

\end{frame}

\begin{frame}[fragile]{Shimko method (9)}

Fun part: plots

\begin{Shaded}
\begin{Highlighting}[]
\NormalTok{shimko.rnd.df =}\StringTok{ }\KeywordTok{data.frame}\NormalTok{(Krange,shimko.rnd.preBrexit, }
\NormalTok{                           shimko.rnd.Brexit, shimko.rnd.postBrexit)}

\KeywordTok{ggplot}\NormalTok{(}\DataTypeTok{data=}\NormalTok{shimko.rnd.df, }\KeywordTok{aes}\NormalTok{(}\DataTypeTok{x=}\NormalTok{Krange)) }\OperatorTok{+}\StringTok{ }
\StringTok{  }\KeywordTok{geom_line}\NormalTok{(}\KeywordTok{aes}\NormalTok{(}\DataTypeTok{y=}\NormalTok{shimko.rnd.preBrexit), }\DataTypeTok{col=}\StringTok{"red"}\NormalTok{, }\DataTypeTok{size=}\FloatTok{1.25}\NormalTok{) }\OperatorTok{+}
\StringTok{  }\KeywordTok{geom_line}\NormalTok{(}\KeywordTok{aes}\NormalTok{(}\DataTypeTok{y=}\NormalTok{shimko.rnd.Brexit), }\DataTypeTok{col=}\StringTok{"darkcyan"}\NormalTok{, }\DataTypeTok{size=}\FloatTok{1.5}\NormalTok{) }\OperatorTok{+}
\StringTok{  }\KeywordTok{geom_line}\NormalTok{(}\KeywordTok{aes}\NormalTok{(}\DataTypeTok{y=}\NormalTok{shimko.rnd.postBrexit), }\DataTypeTok{col=}\StringTok{"blue"}\NormalTok{, }\DataTypeTok{size=}\FloatTok{1.25}\NormalTok{) }\OperatorTok{+}
\StringTok{  }\KeywordTok{geom_vline}\NormalTok{(}\DataTypeTok{xintercept =}\NormalTok{ spot[}\DecValTok{1}\NormalTok{], }\DataTypeTok{col=}\StringTok{"red"}\NormalTok{, }\DataTypeTok{linetype=}\StringTok{"longdash"}\NormalTok{) }\OperatorTok{+}\StringTok{ }
\StringTok{  }\KeywordTok{geom_vline}\NormalTok{(}\DataTypeTok{xintercept =}\NormalTok{ spot[}\DecValTok{2}\NormalTok{], }\DataTypeTok{col=}\StringTok{"darkcyan"}\NormalTok{, }\DataTypeTok{linetype=}\StringTok{"longdash"}\NormalTok{) }\OperatorTok{+}
\StringTok{  }\KeywordTok{geom_vline}\NormalTok{(}\DataTypeTok{xintercept =}\NormalTok{ spot[}\DecValTok{3}\NormalTok{], }\DataTypeTok{col=}\StringTok{"blue"}\NormalTok{, }\DataTypeTok{linetype=}\StringTok{"longdash"}\NormalTok{) }\OperatorTok{+}
\StringTok{  }\KeywordTok{geom_hline}\NormalTok{(}\DataTypeTok{yintercept=}\DecValTok{0}\NormalTok{, }\DataTypeTok{col=}\StringTok{"black"}\NormalTok{, }\DataTypeTok{size=}\FloatTok{0.5}\NormalTok{) }\OperatorTok{+}
\StringTok{  }\KeywordTok{labs}\NormalTok{(}\DataTypeTok{x=}\StringTok{"GBPUSD"}\NormalTok{, }\DataTypeTok{y=}\StringTok{"3-month risk-neutral density"}\NormalTok{)   }
\end{Highlighting}
\end{Shaded}

\end{frame}

\begin{frame}{Shimko method (10)}

\begin{figure}
\includegraphics[width=1\linewidth]{2018_02_07_IMF_FXCourse_files/figure-beamer/unnamed-chunk-76-1} \caption{Shimko risk neutral distributions. Pre-Brexit: red; Brexit: cyan; post-Brexit: blue}\label{fig:unnamed-chunk-76}
\end{figure}

\end{frame}

\begin{frame}{}

\color{blue} \LARGE{Part G:}\\
\LARGE{Comparing distributions}

\end{frame}

\begin{frame}[fragile]{Pre-Brexit (1)}

\begin{Shaded}
\begin{Highlighting}[]
\KeywordTok{ggplot}\NormalTok{(}\DataTypeTok{data=}\NormalTok{shimko.rnd.df, }\KeywordTok{aes}\NormalTok{(}\DataTypeTok{x=}\NormalTok{Krange)) }\OperatorTok{+}\StringTok{ }
\StringTok{  }\KeywordTok{geom_line}\NormalTok{(}\KeywordTok{aes}\NormalTok{(}\DataTypeTok{y=}\NormalTok{shimko.rnd.preBrexit), }\DataTypeTok{col=}\StringTok{"red"}\NormalTok{, }\DataTypeTok{size=}\FloatTok{1.25}\NormalTok{) }\OperatorTok{+}
\StringTok{  }\KeywordTok{geom_line}\NormalTok{(}\DataTypeTok{data=}\NormalTok{gb.rnd.df, }\KeywordTok{aes}\NormalTok{(}\DataTypeTok{x=}\NormalTok{Krange, }\DataTypeTok{y=}\NormalTok{gb.rnd.preBrexit), }
            \DataTypeTok{col=}\StringTok{"darkcyan"}\NormalTok{, }\DataTypeTok{size=}\FloatTok{1.25}\NormalTok{) }\OperatorTok{+}
\StringTok{  }\KeywordTok{geom_line}\NormalTok{(}\DataTypeTok{data=}\NormalTok{ew.rnd.df, }\KeywordTok{aes}\NormalTok{(}\DataTypeTok{x=}\NormalTok{Krange, }\DataTypeTok{y=}\NormalTok{ew.rnd.preBrexit), }
            \DataTypeTok{col=}\StringTok{"blue"}\NormalTok{, }\DataTypeTok{size=}\FloatTok{1.25}\NormalTok{) }\OperatorTok{+}
\StringTok{  }\KeywordTok{geom_hline}\NormalTok{(}\DataTypeTok{yintercept=}\DecValTok{0}\NormalTok{, }\DataTypeTok{col=}\StringTok{"black"}\NormalTok{, }\DataTypeTok{size=}\FloatTok{0.5}\NormalTok{) }\OperatorTok{+}
\StringTok{  }\KeywordTok{labs}\NormalTok{(}\DataTypeTok{x=}\StringTok{"GBPUSD"}\NormalTok{, }\DataTypeTok{y=}\StringTok{"3-month risk-neutral density"}\NormalTok{)   }
\end{Highlighting}
\end{Shaded}

\end{frame}

\begin{frame}{Pre-Brexit (2)}

\begin{figure}
\includegraphics[width=1\linewidth]{2018_02_07_IMF_FXCourse_files/figure-beamer/unnamed-chunk-78-1} \caption{Pre-Brexit risk neutral distributions. Generalized beta: cyan; Edgeworth expansion: blue; Shimko: red.}\label{fig:unnamed-chunk-78}
\end{figure}

\end{frame}

\begin{frame}[fragile]{Brexit (1)}

\begin{Shaded}
\begin{Highlighting}[]
\KeywordTok{ggplot}\NormalTok{(}\DataTypeTok{data=}\NormalTok{shimko.rnd.df, }\KeywordTok{aes}\NormalTok{(}\DataTypeTok{x=}\NormalTok{Krange)) }\OperatorTok{+}\StringTok{ }
\StringTok{  }\KeywordTok{geom_line}\NormalTok{(}\KeywordTok{aes}\NormalTok{(}\DataTypeTok{y=}\NormalTok{shimko.rnd.Brexit), }\DataTypeTok{col=}\StringTok{"red"}\NormalTok{, }\DataTypeTok{size=}\FloatTok{1.25}\NormalTok{) }\OperatorTok{+}
\StringTok{  }\KeywordTok{geom_line}\NormalTok{(}\DataTypeTok{data=}\NormalTok{gb.rnd.df, }\KeywordTok{aes}\NormalTok{(}\DataTypeTok{x=}\NormalTok{Krange, }\DataTypeTok{y=}\NormalTok{gb.rnd.Brexit), }
            \DataTypeTok{col=}\StringTok{"darkcyan"}\NormalTok{, }\DataTypeTok{size=}\FloatTok{1.25}\NormalTok{) }\OperatorTok{+}
\StringTok{  }\KeywordTok{geom_line}\NormalTok{(}\DataTypeTok{data=}\NormalTok{ew.rnd.df, }\KeywordTok{aes}\NormalTok{(}\DataTypeTok{x=}\NormalTok{Krange, }\DataTypeTok{y=}\NormalTok{ew.rnd.Brexit), }
            \DataTypeTok{col=}\StringTok{"blue"}\NormalTok{, }\DataTypeTok{size=}\FloatTok{1.25}\NormalTok{) }\OperatorTok{+}
\StringTok{  }\KeywordTok{geom_hline}\NormalTok{(}\DataTypeTok{yintercept=}\DecValTok{0}\NormalTok{, }\DataTypeTok{col=}\StringTok{"black"}\NormalTok{, }\DataTypeTok{size=}\FloatTok{0.5}\NormalTok{) }\OperatorTok{+}
\StringTok{  }\KeywordTok{labs}\NormalTok{(}\DataTypeTok{x=}\StringTok{"GBPUSD"}\NormalTok{, }\DataTypeTok{y=}\StringTok{"3-month risk-neutral density"}\NormalTok{) }
\end{Highlighting}
\end{Shaded}

\end{frame}

\begin{frame}{Brexit (2)}

\begin{figure}
\includegraphics[width=1\linewidth]{2018_02_07_IMF_FXCourse_files/figure-beamer/unnamed-chunk-80-1} \caption{Brexit risk neutral distributions. Generalized beta: cyan; Edgeworth expansion: blue; Shimko: red.}\label{fig:unnamed-chunk-80}
\end{figure}

\end{frame}

\begin{frame}[fragile]{Post-Brexit (1)}

\begin{Shaded}
\begin{Highlighting}[]
\KeywordTok{ggplot}\NormalTok{(}\DataTypeTok{data=}\NormalTok{shimko.rnd.df, }\KeywordTok{aes}\NormalTok{(}\DataTypeTok{x=}\NormalTok{Krange)) }\OperatorTok{+}\StringTok{ }
\StringTok{  }\KeywordTok{geom_line}\NormalTok{(}\KeywordTok{aes}\NormalTok{(}\DataTypeTok{y=}\NormalTok{shimko.rnd.postBrexit), }\DataTypeTok{col=}\StringTok{"red"}\NormalTok{, }\DataTypeTok{size=}\FloatTok{1.25}\NormalTok{) }\OperatorTok{+}
\StringTok{  }\KeywordTok{geom_line}\NormalTok{(}\DataTypeTok{data=}\NormalTok{gb.rnd.df, }\KeywordTok{aes}\NormalTok{(}\DataTypeTok{x=}\NormalTok{Krange, }\DataTypeTok{y=}\NormalTok{gb.rnd.postBrexit), }
            \DataTypeTok{col=}\StringTok{"darkcyan"}\NormalTok{, }\DataTypeTok{size=}\FloatTok{1.25}\NormalTok{) }\OperatorTok{+}
\StringTok{  }\KeywordTok{geom_line}\NormalTok{(}\DataTypeTok{data=}\NormalTok{ew.rnd.df, }\KeywordTok{aes}\NormalTok{(}\DataTypeTok{x=}\NormalTok{Krange, }\DataTypeTok{y=}\NormalTok{ew.rnd.postBrexit), }
            \DataTypeTok{col=}\StringTok{"blue"}\NormalTok{, }\DataTypeTok{size=}\FloatTok{1.25}\NormalTok{) }\OperatorTok{+}
\StringTok{  }\KeywordTok{geom_hline}\NormalTok{(}\DataTypeTok{yintercept=}\DecValTok{0}\NormalTok{, }\DataTypeTok{col=}\StringTok{"black"}\NormalTok{, }\DataTypeTok{size=}\FloatTok{0.5}\NormalTok{) }\OperatorTok{+}
\StringTok{  }\KeywordTok{labs}\NormalTok{(}\DataTypeTok{x=}\StringTok{"GBPUSD"}\NormalTok{, }\DataTypeTok{y=}\StringTok{"3-month risk-neutral density"}\NormalTok{)   }
\end{Highlighting}
\end{Shaded}

\end{frame}

\begin{frame}{Post-Brexit(2)}

\begin{figure}
\includegraphics[width=1\linewidth]{2018_02_07_IMF_FXCourse_files/figure-beamer/unnamed-chunk-82-1} \caption{Post-Brexit risk neutral distributions. Generalized beta: cyan; Edgeworth expansion: blue.}\label{fig:unnamed-chunk-82}
\end{figure}

\color{blue} \LARGE{Thank You}\\
\LARGE{Have Fun!}

\end{frame}

\end{document}
